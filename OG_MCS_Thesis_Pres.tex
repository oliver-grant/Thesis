\documentclass{beamer}


\usepackage{beamerthemesplit} 
\usepackage{tikz}
\usepackage{algorithm}
\usepackage{algpseudocode}
\usepackage{pifont}
\usepackage{amsmath}
\usepackage{amssymb}
\usepackage{enumerate}
\usepackage{amsfonts}
\usepackage{url}
\usepackage{amsmath,amsfonts,amssymb,amsthm,epsfig,epstopdf,url,array}
\usepackage[]{units}
\usepackage{xcolor}
\usepackage{fancybox}
\usepackage{tikz}
\usepackage{stmaryrd}
\usepackage{caption}
\usepackage{multicol}
\usepackage{hhline}
\usepackage{subcaption}
\usepackage{multirow}
\theoremstyle{plain}
\newtheorem{thm}{Theorem}[section]
\newtheorem{lem}[thm]{Lemma}
\newtheorem{prop}[thm]{Proposition}
\newtheorem*{cor}{Corollary}
\usepackage{color}

\newcommand{\backupbegin}{
   \newcounter{framenumberappendix}
   \setcounter{framenumberappendix}{\value{framenumber}}
}
\newcommand{\backupend}{
   \addtocounter{framenumberappendix}{-\value{framenumber}}
   \addtocounter{framenumber}{\value{framenumberappendix}} 
}

\newcommand*\rfrac[2]{{#1}/{#2}}
\makeatletter
\newcommand{\rom}[1]{\romannumeral #1}

\definecolor{RazerGreen}{RGB}{71,225,12}
\definecolor{ForestGreen}{RGB}{0,100,0}
\definecolor{TruePurple}{RGB}{150,0,150}
\definecolor{AlexBlue}{RGB}{30, 144, 255}
\setbeamercolor{title}{fg=black}
\setbeamercolor{frametitle}{fg=black}
\setbeamercolor{structure}{fg=AlexBlue}
\setbeamertemplate{footline}[frame number]
\beamertemplatenavigationsymbolsempty
\AtBeginSection{\frame{\sectionpage}}

\tikzset{
    current/.style={
		scale=0.8,
		fill=blue,
		minimum size=0.5pt,
        draw,
        circle,
    }
}
\tikzset{
    select/.style={
		scale=0.8,
		fill=red,
		minimum size=0.5pt,
        draw,
        circle,
    }
}
\tikzset{
    max/.style={
		scale=0.8,
		minimum size=0.5pt,
        draw,
        circle,
    }
}

\tikzset{
  treenode/.style = {align=center, inner sep=0pt, text centered,
    font=\sffamily\small},
  arn_n/.style = {treenode, circle, draw=black,
    fill=white, text width=2em},% arbre rouge noir, noeud noir
  arn_small/.style = {treenode, circle, draw=black,
    fill=white, text width=2em, font=\tiny, align=center},
  arn_small_r/.style = {treenode, rectangle, draw,
    fill=white, font=\tiny, align=center},
  arn_r/.style = {treenode, circle, red, draw=red, 
    text width=2em, very thick},% arbre rouge noir, noeud rouge
  arn_x/.style = {treenode, rectangle, draw=black,
    minimum width=2.2em, minimum height=2.5em}% arbre rouge noir, nil
}

\title{Approximately Optimum Search
Trees in External Memory Models}
\subtitle{Master of Mathematics Presentation \\ University of Waterloo}
\author{Oliver Grant \\
Supervisor: J. Ian Munro}



\begin{document} \date{}
\begin{frame}
  \titlepage
\end{frame}

%\tableofcontents

\section{Introduction and Background}

\begin{frame} \frametitle{Binary Search Trees}

\begin{itemize}

\item \textbf{BST} - Simple structure used to store key-value pairs

\item First appeared in the late 1950s/early 1960s

\item Attributed to Windley, Booth, Colin and Hibbard \cite{windley1960trees, booth1960efficiency, hibbard1962some}

\item Ordering property over keys allows for quick searches

\end{itemize}

\end{frame}

\begin{frame} \frametitle{The Optimum Binary Search Tree Problem}

\begin{itemize}

\item Knuth proposed optimum BST problem in 1971 \cite{knuth1971optimum}

\item $n$ ordered keys, {\color{blue}$B_1, B_2, ..., B_n$}

\item Keys must be internal nodes, gaps leaves

\end{itemize}

\begin{center}
\begin{tabular}{ |c|c|c }\cline{1-2}
\cline{1-2}
 \textbf{Word or Gap} & \textbf{Probability} \\
 \cline{1-2}
 $(-\infty, {\color{blue}BE})$ & {\color{ForestGreen}$0.10$} & $\leftarrow$ {\color{ForestGreen} $ q_0$}\\ \cline{1-2}
 {\color{blue}$BE$} & {\color{TruePurple}$0.10$} & $\leftarrow$ {\color{TruePurple} $ p_1$}\\ \cline{1-2}
 $({\color{blue}BE}, {\color{blue}THE})$ & {\color{ForestGreen}$0.40$} & $\leftarrow$ {\color{ForestGreen} $ q_1$}\\ \cline{1-2}
 {\color{blue}$THE$} & {\color{TruePurple}$0.15$} & $\leftarrow$ {\color{TruePurple} $ p_2$}\\ \cline{1-2}
 $({\color{blue}THE}, {\color{blue}TO})$& {\color{ForestGreen}$0.01$} & $\leftarrow$ {\color{ForestGreen} $ q_2$}\\ \cline{1-2}
 {\color{blue}$TO$} & {\color{TruePurple}$0.10$} & $\leftarrow$ {\color{TruePurple} $ p_3$}\\ \cline{1-2}
 $({\color{blue}TO}, \infty)$ & {\color{ForestGreen}0.14} & $\leftarrow$ {\color{ForestGreen} $ q_3$}\\ \cline{1-2}
  & $\mathbf{1.0}$ \\ \cline{1-2}
\end{tabular}
\end{center}
\end{frame}


\begin{frame} \frametitle{Optimum BST Example}

\begin{center}
\begin{tikzpicture}[level/.style={sibling distance = 6cm/####1,
  level distance = 1.5cm}] 
\node (Root) [arn_n] {{\color{TruePurple}$0.15$} {\color{blue}THE}}
    child{ node [arn_n] {{\color{TruePurple}$0.10$}  {\color{blue}$BE$}} 
            child{ node [arn_x] {{\color{ForestGreen}$0.10$} \\ $(-\infty, {\color{blue}BE})$}} 
            	child{ node [arn_x] {{\color{ForestGreen}$0.40$} \\  $({\color{blue}BE}, {\color{blue}THE})$}}
            }
    child{ node [arn_n] {{\color{TruePurple}$0.10$} {\color{blue}$TO$}}
            child{ node [arn_x] {{\color{ForestGreen}$0.01$} \\ $({\color{blue}THE}, {\color{blue}TO})$}}
            child{ node [arn_x] {{\color{ForestGreen}0.14} \\ $({\color{blue}TO}, \infty)$}}
		};
\begin{scope}[xshift=3in,every tree node/.style={},edge from parent path={}]
\path (Root) ++(-6cm,0) node {\textbf{0}};
\path (Root-1) ++(-3cm,0) node {\textbf{1}};
\path (Root-1-1) ++(-1.5cm,0) node {\textbf{2}};
\end{scope}
\end{tikzpicture}




\begin{equation}
C = \sum_{i=1}^{n} {\color{TruePurple}p_i} \cdot (d_T({\color{blue}B_i})+1) + \sum_{i=0}^{n} {\color{ForestGreen}q_i} \cdot d_T \big( ({\color{blue}B_i}, {\color{blue}B_{i+1}}) \big)
\end{equation}
\end{center}
\end{frame}

\begin{frame} \frametitle{Three-Way Branching}

\begin{center}
\begin{tikzpicture}[level/.style={sibling distance = 6cm/####1,
  level distance = 1.5cm}] 
\node (Root) [arn_n] {{\color{TruePurple}$0.15$} {\color{blue}THE}}
    child{ node [arn_x] {{\color{ForestGreen}$0.60$} \\ $(-\infty, {\color{blue}THE})$}
            }
    child{ node [arn_x] {{\color{ForestGreen}$0.25$} \\ $({\color{blue}THE}, \infty)$}
		};
\end{tikzpicture}
\end{center}

\begin{itemize}



\item \definecolor{lightgrey}{rgb}{0.95,0.92,0.92} % Defines the color used for content box headers
\colorbox{lightgrey}{ \fontfamily{cmtt}\selectfont \uppercase{IF (expr) label1, label2, label3} } 

\item Control transferred to one of $3$ locations with one statement

\item $1$ \textit{vs.} $1.4$ comparisons expected using $<, >, =$
\end{itemize}
\end{frame}


\begin{frame}{Overview}

\begin{itemize}

\item We have three separate results:
\begin{enumerate}
\item In the standard problem, we improve bound on MME Heuristic of G{\"u}ttler, Mehlhorn and Schneider \cite{guttler1980binary}

\item Under Hierarchical Memory Model \cite{aggarwal1987model}, two $O(n)$ time approximation algorithms

\item With unordered probabilities, \textit{GREEDY-MS} within $\frac{n+1}{2n}$ of OPT
\end{enumerate}
\end{itemize}
\end{frame}

\begin{frame} \frametitle{Optimum Binary Search Trees}

\begin{itemize}
\item In 1971 C. Gotlieb and Walker approximate solution \cite{walker1971top}

\item Knuth $O(n^2)$ time and space solution \cite{knuth1971optimum}

\item Approximate solutions followed (using less space)

\item Discuss in terms of \textit{entropy}:
\end{itemize}

\begin{align*}
H = \sum_{i=1}^{n} p_i\cdot\lg \left(\frac{1}{p_i} \right) + \sum_{j=0}^{n} q_i\cdot\lg \left( \frac{1}{q_j} \right)
\end{align*}

\end{frame}

\begin{frame} \frametitle{Approximately Optimum Binary Search Trees}


\begin{itemize}
\item In 1975, P. Bayer showed \cite{bayer1975improved}:
\begin{align*}
H-\lg H-(\lg e-1) \leq C_{Opt} \leq C_{WB}, C_{MM} \leq H + 2
\end{align*}

\begin{scriptsize}

\item $C_{Opt}$ optimal solution

\item $C_{WB}, C_{MM}$ both top-down greedy approaches running in $O(n)$

\item $P_L(B_i)$,  $P_R(B_i)$ probs of search before or after $B_i$ respectively

\item $C_{WB}$: Root $B_i$ with minimum $|P_L(B_i)-P_R(B_i)|$

\item $C_{MM}$: Root $B_i$ with minimum $\max(P_L(B_i), P_R(B_i))$

\end{scriptsize}
\end{itemize}
\end{frame}

\begin{frame} \frametitle{Approximately Optimum Binary Search Trees}

\begin{itemize}
\uncover<1->{
\item In 1993 De Prisco and De Santis gave a new heuristic \cite{de1993binary}:
\begin{align*}
C \leq H+1-q_0-q_n+q_{max}
\end{align*}
 \begin{scriptsize}
 {\setlength\itemindent{25pt} \item $q_{max}$ is the maximum} weight leaf node
 \end{scriptsize}}
 \uncover<2->{
\item Updated in 2009 by Bose and Dou\"{i}eb to \cite{bose2009efficient}:
\begin{align*}
C \leq H + 1 - q_0 - q_n + q_{max} - \sum_{i=0}^{m'} pq_{\text{rank}[i]}
\end{align*}
\begin{scriptsize}
{\setlength\itemindent{25pt} \item $m'=\max({2n-3P,P})-1 \geq \frac{n}{2} - 1$}
{\setlength\itemindent{25pt} \item $P$: number of incr. or decr. sequences in the probabilities of the leaves}
{\setlength\itemindent{25pt} \item $pq_{\text{rank}[i]}$ is the $i^{th}$ smallest probability among all except $q_0$ and $q_n$.}
\end{scriptsize}}

\end{itemize}
\end{frame}

\begin{frame} \frametitle{The Modified Minimum Entropy Heuristic}

\begin{itemize}

\item<1-> 1980 G{\"u}ttler, Mehlhorn and Schneider described the method

\item<2-> In addition to $H$ being the entropy of our whole distribution:
\begin{align*}
H(x_1,x_2,...,x_n) = \sum_{i=1}^{n} x_i\cdot\lg \left(\frac{1}{x_i} \right)
\end{align*}

\item<3-> Greedy top-down approach (basic idea):

\item<4-> Selects root $B_i$ such that $H(P_L(B_i), p_i, P_R(B_i))$ is maximized

\item<5-> $O(n^2)$ time, $O(n)$ space

\end{itemize}

\end{frame}

\section{An Improved Bound for the MME Heuristic}\label{An Improved Bound for the Modified Minimum Entropy Heuristic}

\begin{frame} \frametitle{An Improved Bound for the Modified Minimum Entropy Heuristic}

\begin{itemize}
\item<1-> Despite good empirical results, did not have a strong bound.

\item<2-> Previous bound: $C_{MME} \leq \mathbf{c_1}\cdot H+2$
\begin{itemize}
\item<3-> where $\mathbf{c_1}=\frac{1}{H(\frac{1}{3}, \frac{2}{3})} \approx 1.08$
\end{itemize}

\item<4-> We show the expected cost is $C_{MME} \leq H+4$.
\end{itemize}

\end{frame}


\begin{frame} \frametitle{Preliminaries}
\only<1,3,5,6>{
\begin{itemize}
\item<1,3,5,6> For subtree $t$, we let 
\begin{equation}
p_t=\sum_{i : B_i \in t} p_i + \sum_{i : (B_i, B_{i+1}) \in t} q_i
\end{equation}
\item<3,5,6> $P_{L_t}(B_i)$, $P_{R_t}(B_i)$: normalized probabilities of searching before or after $B_i$:
\begin{equation}
P_{L_t}(B_i) = \frac{\sum_{i : B_i \in t} p_i + \sum_{i : (B_i, B_{i+1}) \in t} q_i}{p_t}.
\end{equation}
\item<5,6> $P_{L_t}(B_i,B_{i+1})$ and $P_{R_t}(B_i,B_{i+1})$ analogous. \\
\item<6> The local entropy of subtree $t$ is:
\begin{equation}
E_t=H \left(P_{L_t}(B_i), \frac{p_i}{p_t}, P_{R_t}(B_i) \right)
\end{equation} 
\end{itemize}}

\begin{figure}
\begin{tikzpicture}[level/.style={sibling distance = 6cm/####1,
  level distance = 1.5cm}] 
  \only<2,4,7->{
\node (Root) [arn_n] {{\color{TruePurple}$0.15$} {\color{blue}THE}}
    child{ node [arn_n] {{\color{TruePurple}$0.10$}  {\color{blue}$BE$}} 
            child{ node [arn_x] {{\color{ForestGreen}$0.10$} \\ $(-\infty, {\color{blue}BE})$}} 
            	child{ node [arn_x] {{\color{ForestGreen}$0.40$} \\  $({\color{blue}BE}, {\color{blue}THE})$}}
            }
    child{ node [arn_n] {{\color{TruePurple}$0.10$} {\color{blue}$TO$}}
            child{ node [arn_x] {{\color{ForestGreen}$0.01$} \\ $({\color{blue}THE}, {\color{blue}TO})$}}
            child{ node [arn_x] {{\color{ForestGreen}0.14} \\ $({\color{blue}TO}, \infty)$}}
		}};
\end{tikzpicture}
\end{figure}

\uncover<8->{
\begin{align*}
E_{BE} = H \left( \frac{1}{6}, \frac{1}{6}, \frac{4}{6} \right)
\end{align*}}

\end{frame}


\begin{frame} \frametitle{The Entropy Rule}\label{The Entropy Rule}

\begin{itemize}
\item The basic version of the heuristic greedily chooses root in a top-down manner:
\end{itemize}
\begin{center}
\textit{$B_i$ is selected such that $H \left( P_{L_t}(B_i), \frac{p_i}{p_t}, P_{R_t}(B_i) \right)$ is maximized.}
\end{center}

\begin{itemize}
\item Fundamentally, we are getting a "good" three-way split.
\end{itemize}

\end{frame}

\begin{frame} \frametitle{The Entropy Rule's Shortcomings}

\begin{figure}[H]
\centering
% Set the overall layout of the tree
\tikzstyle{level 1}=[level distance=3cm, sibling distance=3cm]
\tikzstyle{level 2}=[level distance=3cm, sibling distance=3cm]
\scriptsize
\begin{subfigure}{.46\textwidth}
\centering
\begin{tikzpicture}[scale=0.5]
\node [arn_n] {\color{TruePurple}$0.2$}
  child {node [arn_x] {\color{ForestGreen}$0.2$} }  
  child {node [arn_n] {\color{TruePurple}$0.0$} 
	  child {node [arn_x] {\color{ForestGreen}$0.0$}}
	  child {node [arn_x] {\color{ForestGreen}$0.6$}}
  };
\end{tikzpicture}
\caption{Entropy rule tree: $C=1.6$}
\end{subfigure}
\begin{subfigure}{.46\textwidth}
\centering
\begin{tikzpicture}[scale=0.5]
\node [arn_n] {\color{TruePurple}$0.0$}
  child {node [arn_n] {\color{TruePurple}$0.2$} 
	  child {node [arn_x] {\color{ForestGreen}$0.2$}}
	  child {node [arn_x] {\color{ForestGreen}$0.0$}}
  }
  child {node [arn_x] {\color{ForestGreen}$0.6$} }
  ;
\end{tikzpicture}
\caption{MME rule tree: $C=1.4$}
\end{subfigure}
\end{figure}

We note that $H(0.2, 0.2, 0.6) > H(0.4, 0.0, 0.6)$ so, naively we would choose the left tree.

\end{frame}

\begin{frame} \frametitle{The Modified Minimum Entropy Rule}

\only<1,3,5,7>{
\begin{itemize}
\item<1,3,5,7>[\textit{a)}] If there exists key $B_i$ such that $\frac{p_i}{p_t} > \max(P_{L_t}(B_i), P_{R_t}(B_i))$ we always select $B_i$ as the root.

\item<3,5,7>[\textit{b)}] If there exists a gap $(B_i, B_{i+1})$ such that $\frac{q_i}{p_t} > max(P_{L_t}(B_i, B_{i+1}), P_{R_t}(B_i, B_{i+1}))$ then we select the root from among $B_i$ and $B_{i+1}$. $B_i$ is chosen if $P_{L_t}(B_i, B_{i+1}) > P_{R_t}(B_i, B_{i+1})$ and $B_{i+1}$ is chosen otherwise.

\item<5,7>[\textit{c)}] Otherwise, $B_i$ is selected such that $H(P_{L_t}(B_i), \frac{p_i}{p_t}, P_{R_t}(B_i))$ is maximized (as in the original entropy rule).

\end{itemize}}
\only<7>{The approach proposed by G{\"u}ttler, Mehlhorn and Schneider takes $O(n^2)$ time in the worst case and $O(n)$ space.}

\only<2>{
\begin{figure}
\begin{tikzpicture}
  \only<2>{
  \draw[color=TruePurple] (0,-0.5) rectangle (1,0.5) node[pos=.5] {$0.1$};
  \draw[color=ForestGreen] (1.5,0) ellipse (0.5cm and 0.5cm) node[align=center] {$0.1$};
  \draw[color=TruePurple] (2,-0.5) rectangle (3,0.5) node[pos=.5] {$0.1$};
  \draw[color=ForestGreen,thick] (5,0) ellipse (2cm and 0.5cm) node[align=center] {$0.4$};
  \draw[color=TruePurple] (7,-0.5) rectangle (8,0.5) node[pos=.5] {$0.1$};
  \draw[color=ForestGreen] (8.5,0) ellipse (0.5cm and 0.5cm) node[align=center] {$0.1$};
  \draw[color=TruePurple] (9,-0.5) rectangle (10,0.5) node[pos=.5] {$0.1$};
}
\end{tikzpicture}
\end{figure}}

\only<4>{
\begin{figure}
\begin{tikzpicture}
  \only<4>{
  \draw[color=TruePurple] (0,-0.5) rectangle (1,0.5) node[pos=.5] {$0.11$};
  \draw[color=ForestGreen, thick] (1.5,0) ellipse (0.5cm and 0.5cm) node[align=center] {$0.1$};
  \draw[color=TruePurple] (2,-0.5) rectangle (8,0.5) node[pos=.5] {$0.6$};
  \draw[color=ForestGreen] (8.5,0) ellipse (0.5cm and 0.5cm) node[align=center] {$0.1$};
  \draw[color=TruePurple] (9,-0.5) rectangle (10,0.5) node[pos=.5] {$0.09$};
}
\end{tikzpicture}
\end{figure}}


\begin{figure}
\begin{tikzpicture}
  \only<6>{
  \draw[color=TruePurple] (0,-0.5) rectangle (1,0.5) node[pos=.5] {$0.1$};
  \draw[color=ForestGreen,thick] (3,0) ellipse (2cm and 0.5cm) node[align=center] {$0.4$};
  \draw[color=TruePurple] (5,-0.5) rectangle (6,0.5) node[pos=.5] {$0.1$};
  \draw[color=ForestGreen] (6.5,0) ellipse (0.5cm and 0.5cm) node[align=center] {$0.1$};
  \draw[color=TruePurple] (7,-0.5) rectangle (8,0.5) node[pos=.5] {$0.1$};
  \draw[color=ForestGreen] (8.5,0) ellipse (0.5cm and 0.5cm) node[align=center] {$0.1$};
  \draw[color=TruePurple] (9,-0.5) rectangle (10,0.5) node[pos=.5] {$0.1$};
}
\end{tikzpicture}
\end{figure}


\end{frame}


\begin{frame} \frametitle{MME is Within 4 of Entropy}

\uncover<1->{
\begin{itemize}
\item We start with two important equations:
\end{itemize}
\begin{align*}
C_{MME} &= \sum_{t \in S_T} p_t
 &H = \sum_{t \in S_T} p_t \cdot E_t
\end{align*}  
where $S_T$ is the set of all subtrees of our tree $T$. \\
}
\uncover<2->{
\noindent For each subtree, we bind $E_t \geq 1 - \frac{b_t}{p_t}$ (we define $b_t$ later): }

\begin{align*}
\uncover<3->{H &= \sum_{t \in S_T} p_t E_t \scriptsize{\text{ (sub in } E_t \text{ bound)}} \\}
\uncover<4->{
&\geq \sum_{t \in S_T} p_t \cdot \left(1 - \frac{b_t}{p_t} \right) = C_{MME} - \sum_{t \in S_T} b_t \\}
\uncover<5->{
 \implies C_{MME} &\leq H + \sum_{t \in S_T} b_t}
\end{align*}

\end{frame}

\begin{frame}

\begin{itemize}
\item<1-> For each root, we show that one of three cases must occur:

\item<2->[\textit{Case 1)}] If there is a "big key" OR there is a key spanning the middle: $E_t \geq 1-2 \frac{p_r}{p_t}$. 
\begin{itemize}
\item \textbf{We set $b_t = 2p_r$.} ($p_r$ is the middle root)
\item This can only happen \textbf{once for each root}.
\end{itemize}

\item<3->[\textit{Case 2)}] If there is a "big gap" with prob $q_i$ we get: $E_t \geq 1-\frac{q_i}{p_t}$ 
\begin{itemize}
\item \textbf{We set $b_t = q_i$.}
\item This can happen at most \textbf{twice for each gap}.
\end{itemize}

\item<4->[\textit{Case 3)}] Otherwise we do manipulations and show our root choice cannot be "too far" from the centre and get $E_t \geq 1 - \frac{4}{5} \cdot \left( \frac{q_i}{p_t}\right)^2$
\begin{itemize}
\item $q_i$ is the middle-spanning gap
\item \textbf{We set $b_t = \frac{4}{5} \cdot \frac{q_i^2}{p_t}$.}
\item Through careful manipulations we show that summing over all cases for a single \textbf{$q_i$ can sum to at most $2\cdot q_i$}.
\end{itemize} 
\end{itemize}
\end{frame}

\begin{frame}{}
\begin{itemize}
\item[\textit{Case 1)}] $b_t = 2p_r$, \textbf{once for each root}.

\item[\textit{Case 2)}]$b_t = q_i$, \textbf{twice for each gap}.

\item[\textit{Case 3)}] $b_t = \frac{q_i^2}{p_t}$, \textbf{all of these cases summed for $q_i$: at most $2\cdot q_i$}.
\end{itemize}

Combining the above we get that our cost is at most:
\begin{align*}
C &\leq H + \sum_{t \in S_T} b_t \\
 &\leq H + 2 \cdot \sum_{i=1}^{n} p_i + 4 \cdot \sum_{j=0}^{n} q_i \\
 &\leq H + 4
\end{align*}
\end{frame}

\section{Approximate Binary Search in the Hierarchical Memory Model}\label{Approximate Binary Search in the Hierarchical Memory Model} 

\begin{frame} \frametitle{Optimum Binary Search Trees on the HMM Model}
\begin{itemize}
\item Given $n$ keys and $n+1$ gaps with associated probabilities.  
\item Construct binary tree $T$ (as before).
\item Assign the nodes in $T$ to memory locations
\item Minimize the expected cost of search.
\end{itemize}
\end{frame}
 
\begin{frame} \frametitle{The Hierarchical Memory Model}\label{The Hierarchical Memory Model}

\begin{multicols}{2}
\begin{itemize}

\item HMM, Aggarwal et al. 1987
\item Not perfect, improvement over the RAM model
\item Hierarchy of different memories: increasing size and increasing access cost

\end{itemize}
\columnbreak

\begin{tiny}
\begin{center}
\begin{tabular}{ |c|c|c}
\cline{1-2}
 \textbf{Register} & \textbf{Data} \\\cline{1-2}
 $1$ & & $\leftarrow c_1$ \\\cline{1-2}
 $2$ & & $\leftarrow c_1$ \\\cline{1-2}
 $...$ & &  \\\cline{1-2}
 $m_1$ & & $\leftarrow c_1$ \\ \hhline{==~}
 
 $m_1 + 1$ & & $\leftarrow c_2$ \\\cline{1-2}
 $m_1 + 2$ & & $\leftarrow c_2$ \\\cline{1-2}
 . & &  \\\cline{1-2}
 . & &  \\\cline{1-2}
 . & &  \\\cline{1-2}
 $m_1+m_2=m_2'$ & & $\leftarrow c_2$ \\ \hhline{==~}
 $m_1+m_2 + 1$ & & $\leftarrow c_3$ \\\cline{1-2}
 $m_1+m_2 + 2$ & & $\leftarrow c_3$ \\\cline{1-2}
 . & &  \\\cline{1-2}
 . & &  \\\cline{1-2}
 . & &  \\\cline{1-2}
 . & &  \\\cline{1-2}
 . & &  \\\cline{1-2}
 $m_1+m_2+m_3=m_3'$ & & $\leftarrow c_3$ \\ \hhline{==~}
 ... & 
\end{tabular}
\end{center}
\end{tiny}
\end{multicols}
\end{frame}

\begin{frame} \frametitle{Multiway/K-ary trees} \label{sec:MWT}
\begin{tiny}
\begin{center}

\begin{tikzpicture}[scale=0.9]

\node (Root) [circle,draw] (z){2\: 10 12}
  child {node [rectangle,draw] (a) {$x<2$}    
  }  
  child {node [circle,draw] (b) {4\: \:6\: \:8}
    child {node [rectangle,draw] (a) {$2<x<4$}}
    child {node [rectangle,draw] (a) {$4<x<6$}}
    child {node [rectangle,draw] (a) {$6<x<8$}}
    child {node [rectangle,draw] (a) {$8<x<10$}}
  }  
  child {node [rectangle,draw] (c) {$10<x<12$}
  }
  child {node [rectangle,draw] (d) {$x>12$}
  };
\begin{scope}[xshift=3in,every tree node/.style={},edge from parent path={}]
\path (Root) ++(-4.2cm,0) node {\textbf{0}};
\path (Root-1) ++(-0.8cm,0.18) node {\textbf{1}};
\path (Root-1-1) ++(0.85,0.25) node {\textbf{2}};
\end{scope}
\end{tikzpicture}

\end{center}
\end{tiny}

\begin{itemize}

\item Given $n$ keys and $n+1$ gaps with probabilities

\item Up to $k-1$ keys in internal node, single gap in a leaf node

\item Cost of search within an internal node is constant
\begin{align*}
C =\sum_{i=1}^{n} p_i \cdot (d_T(B_i)+1) + \sum_{j=0}^{n} q_j \cdot d_T \big((B_{i-1},B_i) \big)
\end{align*}


\end{itemize}

\end{frame}

\begin{frame} \frametitle{Back to: Optimum Binary Search Trees on the HMM Model}
\begin{itemize}
\item Given $n$ keys and $n+1$ gaps with associated probabilities
\item Construct binary tree $T$ (as before)
\item Assign the nodes in $T$ to memory locations
\item Minimize the expected cost of search:
\uncover<2->{
\begin{align*}
C =\sum_{i=1}^{n} p_i \cdot C(B_i) + \sum_{i=0}^{n} q_i \cdot C\big( (B_{i-1}, B_i) \big)
\end{align*}}
\end{itemize}
\end{frame}

\begin{frame}{The Expected Cost of Search}
\begin{multicols}{2}
 \begin{tikzpicture}[level/.style={sibling distance = 3cm/####1,
  level distance = 1cm}]
\node (Root) [arn_n] {\tiny {\color{TruePurple}$0.15$} {\color{blue}THE}}
    child{ node [arn_n] {\tiny {\color{TruePurple}$0.10$}  {\color{blue}$BE$}} 
            child{ node [arn_x] {\tiny {\color{ForestGreen}$0.10$} \\ {\tiny $(-\infty, {\color{blue}BE})$}}} 
            	child{ node [arn_x] {\tiny {\color{ForestGreen}$0.40$} \\ {\tiny $({\color{blue}BE}, {\color{blue}THE})$}}}
            }
    child{ node [arn_n] {\tiny {\color{TruePurple}$0.10$} {\color{blue}$TO$}}
            child{ node [arn_x] {\tiny {\color{ForestGreen}$0.01$} \\ {\tiny $({\color{blue}THE}, {\color{blue}TO})$}}}
            child{ node [arn_x] {\tiny {\color{ForestGreen}0.14} \\ {\tiny $({\color{blue}TO}, \infty)$}}}
		};
\end{tikzpicture}
\columnbreak

\begin{scriptsize}
\begin{center}
\begin{tabular}{|c|c|}
\hline
 \textbf{Register} & \textbf{Data} \\\hline
 $1$ & THE \\\cline{1-2}
 $2$ & $(THE, TO)$ \\\hline\hline
 $3$ & TO \\\cline{1-2}
 $4$ & (TO, $\infty$) \\\cline{1-2}
 $5$ & ($-\infty$, BE) \\\hline\hline
 $6$ & BE \\\cline{1-2}
 $7$ & (BE, THE) \\\cline{1-2}
 . &  \\\cline{1-2}
 . &  \\\cline{1-2}
\end{tabular}
\end{center}
\end{scriptsize}
\end{multicols}

\end{frame}

\begin{frame}[fragile] \frametitle{Algorithm ApproxMWPaging}\label{Algorithm ApproxMWPaging}
\only<1,3,5,7,9>{
\begin{itemize}
\item<1,3,5,7,9>[1.] Create MWT using work of Bose and Dou\"{i}eb, $m_1$ page size

\item<3,5,7,9>[2.] Create balanced BST inside each node (page)

\item<5,7,9>[3.] Connect the balanced BSTs to form a single BST

\item<7,9>[4.] Pack into memory in BFS order

\end{itemize}}

\only<9>{\noindent Left with a BST in our memory in $O(n)$.}

\only<2> {
\begin{tiny}
\begin{center}
\begin{tikzpicture}[scale=0.9]
\node [circle,draw] (z){2\: 10 12}
  child {node [rectangle,draw] (a) {$x<2$}    
  }  
  child {node [circle,draw] (b) {4\: \:6\: \:8}
    child {node [rectangle,draw] (a) {$2<x<4$}}
    child {node [rectangle,draw] (a) {$4<x<6$}}
    child {node [rectangle,draw] (a) {$6<x<8$}}
    child {node [rectangle,draw] (a) {$8<x<10$}}
  }  
  child {node [rectangle,draw] (c) {$10<x<12$}
  }
  child {node [rectangle,draw] (d) {$x>12$}
  };
\end{tikzpicture}
\end{center}
\end{tiny}
}

\only<4> {
% Set the overall layout of the tree
\tikzstyle{level 1}=[level distance=3cm, sibling distance=3.5cm]
\tikzstyle{level 2}=[level distance=3cm, sibling distance=3.5cm]

\tikzset{smnode/.style={level 1/.style={level distance=1cm, sibling distance=1cm}, circle, draw}}
\tikzstyle{level 1}=[level distance=3cm, sibling distance=3.5cm]
\tikzstyle{level 2}=[level distance=3cm, sibling distance=3.5cm]
\tikzstyle{level 3}=[level distance=1cm, sibling distance=1cm]

\begin{tiny}
\begin{center}

\begin{tikzpicture}[scale=0.5]

\node [circle,draw] (z){
  \tikzstyle{level 1}=[level distance=0.5cm, sibling distance=0.5cm]
	\begin{tikzpicture}
	\node [draw,circle] (zZ){10}
  	child {node [draw,circle] (Za) {2}    
  	}  
  	child {node [draw,circle] (Zb) {12}
  	};
	\end{tikzpicture}
}
  child {node [rectangle,draw] (a) {$x<2$}    
  }  
  child {node [circle,draw] (b) {
    \tikzstyle{level 2}=[level distance=0.5cm, sibling distance=0.5cm]
	\begin{tikzpicture}
	\node [draw,circle] (zZa){6}
  	child {node [draw,circle] (Zaa) {4}    
  	}  
  	child {node [draw,circle] (Zba) {8}
  	};
	\end{tikzpicture}  
  }
    child {node [rectangle,draw] (a) {$2<x<4$}}
    child {node [rectangle,draw] (a) {$4<x<6$}}
    child {node [rectangle,draw] (a) {$6<x<8$}}
    child {node [rectangle,draw] (a) {$8<x<10$}}
  }  
  child {node [rectangle,draw] (c) {$10<x<12$}
  }
  child {node [rectangle,draw] (d) {$x>12$}
  };

\end{tikzpicture}

\end{center}
\end{tiny}
}

\only<6>{
\begin{tiny}
\begin{center}
\begin{tikzpicture}[
  scale = 0.6,
  level 1/.style = {sibling distance=10cm},
  level 2/.style = {sibling distance=6cm}, 
  level 3/.style = {sibling distance=6cm},
  level 4/.style = {sibling distance=3cm}
]
\node [arn_small] {10}
  child { node [arn_small] {2} 
    child {node [rectangle,draw] {$x<2$}    
    }
    child {node [arn_small] {6}
      child {node [arn_small] {4}
        child {node [rectangle,draw]{$2<x<4$}}
        child {node [rectangle,draw]{$4<x<6$}}    
      }
      child {node [arn_small]{8}
        child {node [rectangle,draw]{$6<x<8$}}
        child {node [rectangle,draw]{$8<x<10$}}    
      }
    }   
  }
  child {node [arn_small] {12}    
    child {node [rectangle,draw] {$10<x<12$}
    }
    child {node [rectangle,draw] {$x>12$}
    }  
  };
\end{tikzpicture}
\end{center}
\end{tiny}
}

\only<8>{

\begin{tiny}
\begin{center}
    \begin{tabular}{ | l | l | l | l |}
    \hline
    Memory Location & Node & Left Child Location & Right Child Location \\ \hline
    1  & 10  & 2    & 3    \\ \hline
    2  & 2  & 4    & 5    \\ \hline
    3  & 12 & 6    & 7    \\ \hline
    4  & $x < 2$  & ---   & ---    \\ \hline
    5  & 6  & 8   & 9   \\ \hline
    6  & $10 < x < 12$ & ---   & ---   \\ \hline
    7  & $x > 12$ & ---   & ---   \\ \hline
    8  & 4  & 10 & 11 \\ \hline
    9  & 8  & 12 & 13 \\ \hline
    10 & $2 < x < 4$  & --- & --- \\ \hline
    11 & $4 < x < 6$  & --- & --- \\ \hline
    12 & $6 < x < 8$  & --- & --- \\ \hline
    13 & $8 < x < 10$ & --- & 16   \\ \hline
    \end{tabular}
\end{center}

\end{tiny}
}

\end{frame}

\begin{frame} \frametitle{Expected Cost ApproxMWPaging} \label{45}
\begin{itemize}
\item[1.] Bind depth of each key/gap in MWT from its probability
\begin{itemize}
\item Directly from work of Bose and Dou\"{i}eb \cite{bose2009efficient}
\end{itemize}
\item[2.] Bind depth of each key/gap based on its probability in BST
\begin{itemize}
\item Each key has depth at most $\lg(m_1)$ within a page.
\end{itemize}
\item[3.] We bind the location in the memory of each key/gap 
\begin{itemize}
\item Since we use BFS this is straightforward
\end{itemize}
\item[4.] We can then explicitly bound the expected cost of search of our tree.
\end{itemize}
\end{frame}

\begin{frame}\frametitle{Cost in terms of Entropy (Theorem 4.5.8)}
\begin{align*}
C &< \left(H + 1 + \sum_{i=0}^n q_i - q_0 - q_n - \sum_{i=0}^m q_{\text{rank}[i]} \right) \cdot  c_h
\end{align*}
where $q_{\text{rank}[i]}$ is the $i^{th}$ smallest access probability among gaps except $q_0$ and $q_n$.
\begin{proof}
\begin{itemize}
\item Can bound cost of search under standard model.
\item Increasing costs as you move down in the tree.
\item Costs strictly less than $i \cdot c_h$ to access a node at depth $i$
\end{itemize}
\end{proof}
\end{frame}


\begin{frame} \frametitle{Algorithm ApproxBSTPaging}
Our second solution to create an approximately optimal BST under the HMM model works as follows: \\

\begin{itemize}
\item Create a BST $T$ with an algorithm of De Prisco and De Santis \cite{de1993binary} as updated by Bose and Dou\"{i}eb \cite{bose2009efficient}. \\

\item Pack keys from $T$ into memory in a BFS order.


\item Both steps take $O(n)$ time.
\end{itemize}

\end{frame}

\begin{frame} \frametitle{Expected Cost ApproxBSTPaging}\label{48}
\begin{itemize}
\item[1.] Bind depth of each key/gap based on its probability in BST
\begin{itemize}
\item From work of Bose and Dou\"{i}eb \cite{bose2009efficient}
\end{itemize}
\item[2.] We bind the location in the memory of each key/gap 
\begin{itemize}
\item As before, since we use BFS this is straightforward
\end{itemize}
\item[3.] We can then explicitly bound the expected cost of search of our tree.
\end{itemize}
\end{frame}

\begin{frame}\frametitle{Cost in terms of Entropy (Theorem 4.8.5)}
\begin{align*}
C &<  \left(H + 1 - q_0 - q_n + q_{max} - \sum_{i=0}^{m'} pq_{\text{rank}[i]} \right)\cdot c_h
\end{align*}
\begin{proof}
\begin{itemize}
\item Can bound cost of search under standard model.
\item Increasing costs as you move down in the tree.
\item Costs strictly less than $i \cdot c_h$ to access a node at depth $i$
\end{itemize}
\end{proof}
\end{frame}


%=========================================
\section{Binary Trees on Unordered Sequences of Probabilities}\label{BST over Multisets}

\begin{frame} \frametitle{Binary Trees on Unordered Sequences of Probabilities}
We have a multiset of $n$ probabilities (n is odd): \\
$M = \lbag p_1, p_2, ..., p_n \rbag$ such that $\sum\limits_{i=1}^n p_i = 1$.  \\
The cost under \textit{\textbf{S}tandard Model}:
\begin{equation}
C_{T_S} = \sum_{i=1}^{n} p_i(b_i+1) - \sum_{p_i \in L_T} p_i
\end{equation}
$b_i$ is the depth of the node representing probability $p_i$.
\end{frame}

\begin{frame} \frametitle{Binary Trees on Unordered Sequences of Probabilities}
\begin{itemize}
\item Similar to Huffman code problem but we can place probabilities at internal \textbf{and} external node locations.
\item As in previous models, we don't charge to "examine" the leaves.
\item The above constraint makes finding the optimal solution much more difficult.
\item We show our GREEDY-MS Algorithm is within $\frac{n+1}{2n}$ of Optimal.
\end{itemize}
\end{frame}



\begin{frame} \frametitle{The \textbf{E}xpected Path Length Model}
Consider a new cost model for this problem, the \textit{\textbf{E}xpected Path Length Model} (EPL), which has cost $C_{T_E}$ defined as follows:
\begin{equation}
C_{T_E} = \sum_{i=1}^{n} p_i(b_i+1)
\end{equation}
This model treats internal and external nodes the same.
\end{frame}

\begin{frame} \frametitle{GREEDY-MS}
There is a simple, greedy, $O(n \lg n)$ time algorithm that is optimal under EPL \cite{golin2012huffman}. We call this algorithm \textit{GREEDY-MS}:

\begin{itemize}
\item[1.] Sort the multiset $M$ to create $R$.

\item[2.] The root of our tree will be $R[0]$, its two children will be $R[1]$ and $R[2]$, and so on. $R[i]$ will be placed at location $i+1$ in BFS order.
\end{itemize}
\end{frame}

\begin{frame}\frametitle{GREEDY-MS optimal under EPL (Lemma 5.2.1)}
\uncover<1->{
\textbf{Claim 1.}\label{Claim-EPL}
In an optimal tree under EPL, for a given depth $d$, the probabilities of all nodes on level $d+1$ are \textbf{less than or equal to} the probabilities of all nodes on level $d$.}
\uncover<2->{
\begin{proof}
Otherwise we could swap contradicting probabilities and get a better tree.
\end{proof}}
\uncover<3->{
\noindent \textbf{Claim 2.}\label{Claim-EPL2}
In an optimal tree under EPL, $T$ must be full at all depths strictly less than its height.}
\uncover<4->{
\begin{proof}
Otherwise we could place a leaf at an non-full level and get a better tree.
\end{proof}}
\uncover<5->{
Taking claims 1 and 2 together gives the result.}
\end{frame}

\begin{frame}{GREEDY-MS within $\frac{n+1}{2n}$ of optimal under \textit{Standard Model} }
\begin{itemize}
\item[1.] We show that in any BST with all parent probs $\geq$ their children's probs, all but the smallest leaf can be \textit{covered}.
\item[2.] We show that in any BST with all parent probs $\geq$ their children's probs, its leaves have total probability at most $\frac{n+1}{2n}$.
\item[3.] We show that if a tree existed which was more than  $\frac{n+1}{2n}$ better than GREEDY-MS under the \textit{Standard Model}, we would get a contradiction.
\end{itemize}
\end{frame}

\begin{frame}{Covering Lemma (5.2.2)}

\textbf{Definition} \\
An internal node with probability $p_i$ \textit{covers} a node $p_j$ if $p_i \geq p_j$.\\

\begin{lem}
For a BST $T$ with all parent probabilities $\geq$ than their children's:
\begin{align*}
\forall_{p_i \in (L_T-\{l_{min}\})} \exists \text{ unique } p_i' \notin (L_T \cup \{l_{min}\}) \text{ such that } p_i' \text{ covers } p_i
\end{align*}
\end{lem}

The proof follows an inductive argument on the height of the tree.
\end{frame}

\begin{frame}{Covering Lemma (5.2.2) Proof}
In the base case we can always cover the bigger leaf since our parent is bigger.

\begin{center}
\begin{tikzpicture}[level/.style={sibling distance = 6cm/####1,
  level distance = 1.5cm}] 
\node (Root) [arn_n, fill=red!50] {$0.40$ \\ THE}
    child{ node [arn_x, fill=red!50] {$0.35$ \\ $(-\infty, THE)$}
            }
    child{ node [arn_x] {$0.25$ \\ $(THE, \infty)$}
		};
\end{tikzpicture}
\end{center}
\end{frame}

\begin{frame}{Covering Lemma (5.2.2) Proof}
In the inductive step our new root can cover the larger uncovered child.

\begin{center}
\begin{tikzpicture}[level/.style={sibling distance = 6cm/####1,
  level distance = 1.5cm}]
  \only<1>{
\node (Root) [arn_n] {$0.30$ \\ THE}
    child{ node [arn_n,  fill=red!50] {$0.20$ \\  $BE$} edge from parent[draw=none]
            child{ node [arn_x,  fill=red!50] {$0.10$ \\ $(-\infty, BE)$}} 
            	child{ node [arn_x] {$0.06$ \\  $(BE, THE)$}}
            }
    child{ node [arn_n, fill=red!50] {$0.20$ \\ $TO$} edge from parent[draw=none]
            child{ node [arn_x, fill=red!50] {$0.10$ \\ $(THE,TO)$}}
            child{ node [arn_x] {$0.04$ \\ $(TO, \infty)$}}
		}};
		\only<2>{
\node (Root) [arn_n, fill=red!50] {$0.30$ \\ THE}
    child{ node [arn_n, fill=red!50] {$0.20$ \\  $BE$} 
            child{ node [arn_x, fill=red!50] {$0.10$ \\ $(-\infty, BE)$}} 
            	child{ node [arn_x, fill=red!50] {$0.06$ \\  $(BE, THE)$}}
            }
    child{ node [arn_n, fill=red!50] {$0.20$ \\ $TO$}
            child{ node [arn_x, fill=red!50] {$0.10$ \\ $(THE,TO)$}}
            child{ node [arn_x] {$0.04$ \\ $(TO, \infty)$}}
		}};
\end{tikzpicture}

\end{center}

\end{frame}

\begin{frame}{Bound of Leaf probabilities (Lemma 5.2.3)}
For a BST $T$ with all parent probabilities $\geq$ than their children's:
\begin{align*}
\sum_{p_i \in L_T} p_i \leq \frac{n+1}{2n}
\end{align*}
\begin{proof}
Since each leaf except the smallest is covered:
\begin{align*}
\sum_{p_i \in L_T} p_i \leq \frac{1}{2} \cdot (1-\ell_{min}) + \ell_{min}
\end{align*}
$\ell_{min}$ is at most $\frac{1}{n}$ so we get the desired result.
\end{proof}
\end{frame}

\begin{frame}{GREEDY-MS within $\frac{n+1}{2n}$ of optimal under \textit{Standard Model}}
Let $T$ be our tree made by GREEDY-MS. If there existed a tree $T'$ which had cost:
\begin{align*}
C_{T'_S} < C_{T_S} - \frac{n+1}{2n}
\end{align*}
Using the previous Lemma, we would contradict the optimality of $T$ under the EPL Model.
\end{frame}


%=========================================



\section{Conclusion and Open Problems} \label{Conclusion and Open Problems}

\begin{frame} \frametitle{Conclusion}
\begin{itemize}
\item Modified Entropy Rule first proposed by  G{\"u}ttler, Mehlhorn and Schneider in 1980 had a worst case expected cost of $H+4$ (previous $c\cdot H+2$ where $c \approx 1.08$) \cite{guttler1980binary}.

\item ApproxMWPaging and ApproxBSTPaging solve the problem "well" under the HMM in $O(n)$.

\item GREEDY-MS is within $\frac{n+1}{2n}$ when given unordered probabilities under the \textit{Standard Model}.   
\end{itemize}

\end{frame}


\begin{frame}{Open Problems}
\begin{itemize}
\item Is the modified entropy rule at most $H+2$?
\item Can ApproxMWPaging and ApproxBSTPaging be extended to more sophisticated models (copying/movement within the hierarchy)?
\item Is there a polynomial time algorithm which solves the unordered problem optimally under the \textit{Standard Model}?
\item Are there better "fast" heuristics for this problem?
\end{itemize}
\end{frame}


\bibliographystyle{abbrv}
{\tiny
\bibliography{uw-ethesis/uw-ethesis}}


\begin{frame} \frametitle{Why Study Binary Search Trees}
\appendix
\backupbegin

\begin{itemize}
\item AVL trees self-balancing BST in 1963, maintains $O(\lg(n))$ height \cite{adelsonvelskii1963algorithm}

\item B-trees/Red-black trees by R. Bayer in 1972 (with later extensions by others) \cite{bayer1972symmetric}

\item  Allen and Munro examined a MTR heuristic \cite{allen1978self}

\item Splay trees of Sleator and Tarjan in 1985, dynamic optimality conjecture \cite{sleator1985self}

\item Tango trees in 2007 by Demaine et al., $O(\lg \lg n)$-competitive \cite{demaine2007dynamic}

\end{itemize}

\end{frame}


\begin{frame} \frametitle{Why Study Binary Search Trees}

\begin{itemize}

\item Binary space partitions in 3D graphics \cite{schumacker1969study, paterson1992optimal}

\item Binary tries in routers and IP lookup structures \cite{song2010building}

\item Programming languages like C++ \texttt{std::map} \cite{CppMap} 

\item Syntax trees during compilation \cite{louden1997compiler}

\item Many more examples

\item Theoretically interesting

\end{itemize}

\end{frame}




\begin{frame} \frametitle{Alphabetic Trees}
\begin{scriptsize}
\begin{center}
\begin{tikzpicture}[scale = 0.6]
\node[arn_small] {}
  child {node [arn_small] {}
    child {node [arn_small] {0.3 BE}}
    child {node [arn_small] {0.4 THE}}
  }  
  child {node [arn_small] {0.3 TO}
  };
\end{tikzpicture}
\end{center}
\end{scriptsize}

\begin{itemize}
\item Given $n$ keys with $\sum_{i=1}^{n} p_i = 1$

\item $n$ keys are leaves, every internal node has two children

\item We wish to minimize expected search cost: $\sum_{i=1}^{i=n} p_i \cdot d_T(B_i)$


\item Similar to Huffman code, but have an ordering over keys \cite{huffman1952method}

\end{itemize}

\end{frame}

\begin{frame}\frametitle{Alphabetic Trees}
\begin{itemize}

\item Independent optimal solutions due to Hu and Tucker \cite{hu1971optimal} and Garsia and Wachs \cite{garsia1977new}: $O(n \lg n)$ time and $O(n)$ space

\item Both went through proof simplifications \cite{knuth1973sorting, hu1973new, hu1979binary} and \cite{kingston1988new}

\item Eventually shown to be equivalent in 1982 by Hu \cite{Hu1982Book} 

\item 1991 Yeung gave $O(n)$ approximation with worst-case expected cost at most: $H + 2 - p_1-p_n$ \cite{yeung1991alphabetic}

\item In 1993 improved by De Prisco and De Santis to \cite{de1993binary}: $H+1-p_1-p_n+p_{max}$

\item Final improvement in 2009 by Bose and Dou\"{i}eb \cite{bose2009efficient}:
$H+1 -p_1-p_n-\sum_{i=0}^m p_{\text{rank}[i]} + p_{max}$


\end{itemize}
\end{frame}

\begin{frame}

\begin{itemize}
\item For each root, we show that one of three cases must occur:

\item[\textit{Case 1)}] $E_t \geq 1-2 \frac{p_r}{p_t}$. \textbf{We set $b_t = 2p_r$.}

\item[\textit{Case 2)}] There exists gap $(B_i, B_{i+1})$ such that $\frac{q_i}{p_t} > max(P_{L_t}(B_i, B_{i+1}), P_{R_t}(B_i, B_{i+1}))$. \textbf{We set $b_t = q_i$.}

\item[\textit{Case 3)}]  $\max(P_{L_t}(B_r), P_{R_t}(B_r)) < \frac{4}{5}$. \textbf{We set $b_t = \frac{4}{5} \cdot \frac{q_m^2}{p_t}$ where $q_m$ is the middle gap.}

\end{itemize}


\end{frame}

\begin{frame}

\textit{Case 1)} $E_t \geq 1-2 \frac{p_r}{p_t}$. \textbf{We set $b_t = 2p_r$.}


\begin{itemize}
\item "The easy case"
\item If we have a key in the middle this will follow.
\item Happens at most once per key.
\end{itemize}
TODO PICTURE?
\end{frame}
\begin{frame}
\textit{Case 2)} There exists gap $(B_i, B_{i+1})$ such that $\frac{q_i}{p_t} > max(P_{L_t}(B_i, B_{i+1}), P_{R_t}(B_i, B_{i+1}))$. \textbf{We set $b_t = q_i$.}
\begin{itemize}
\item Still fairly easy
\item If we have a big gap in the middle of our dataset, we use rule b).
\item Can happen at most twice for a gap.
\end{itemize}
\end{frame}

\begin{frame}
\textit{Case 3)} $\max(P_{L_t}(B_r), P_{R_t}(B_r)) < \frac{4}{5}$. \textbf{We set $b_t = q_m$ where $q_m$ is the middle gap.}

\begin{itemize}
\item The hard case
\item When this happens, since $\max(P_{L_t}(B_r), P_{R_t}(B_r)) < \frac{4}{5}$, our middle gap must get bigger relative to the remaining probability.
\item once it is big enough ($\geq \frac{1}{2}$), case 2 occurs.
\item We can show that $E_t \geq 1 - \frac{4}{5}(\frac{q_m}{p_t})^2$ so we set $b_t = \frac{4}{5}\frac{q_m^2}{p_t}$ 
\item Each gap can contribute at most:
\begin{align*}
q_m \cdot \sum\limits_{x=0}^{\infty} \frac{1}{2} \cdot (\frac{4}{5}) ^ x = \mathbf{2 \cdot q_m}
\end{align*}
\end{itemize}
\end{frame}

\begin{frame}

\begin{itemize}
\item Combining these three cases we get a total of:
\end{itemize}

\begin{align*}
C &\leq H + \sum_{t \in S_T} b_t = H + 2 \sum\limits_{r = 1}^n p_r + 2 \sum\limits_{m = 0}^n q_m + \sum\limits_{m = 0}^n \sum\limits_{t \in S_m} \frac{4}{5}\frac{q_m^2}{p_t}\\
C &\leq H + 2 + \sum\limits_{m = 0}^n (\frac{4}{5} \cdot q_m) \sum\limits_{x=0}^{\infty} \frac{1}{2} \cdot (\frac{4}{5}) ^ x \\
C &\leq H + 2 + 2 \\
C &\leq H + 4
\end{align*}
\end{frame}

\begin{frame}\frametitle{MWT Depth Bound}
Let $T'$ be the MWT created in Phase 1. For page size $m_1$ we have \cite{bose2009efficient}:
\begin{align*} d_{T'}(B_i) &\leq \lfloor log_{m_1}(\frac{1}{p_i}) \rfloor \\
d_{T'}((B_{i-1},B_i)) &\leq \lfloor log_{m_1}(\frac{2}{q_i}) \rfloor + 1 \text{ for all gaps, and}\\
d_{T'}((B_{i-1},B_i)) &\leq \lfloor log_{m_1}(\frac{1}{q_i}) \rfloor + 1 \\ 
&\text{ for at least $m$ of them (and the two extremal gaps).}
 \end{align*}
\end{frame}

\begin{frame}\frametitle{BST Depth Bound (Lemma 4.5.1)}
Let $T$ be the BST made in Phase 3.
\begin{align*} 
d_T(B_i) &\leq \lg \left(\frac{1}{p_i}\right) \\
d_T((B_{i-1},B_i)) &\leq \lg \left(\frac{1}{q_i} \right) + 2 \text{ for all gaps, and} \\
d_T((B_{i-1},B_i)) &\leq \lg \left(\frac{1}{q_i} \right) + 1 \\
&\text{for at least  $m$ of them (and the two extremal gaps).}
\end{align*}
\begin{proof}
Depth within each page at most $\lfloor \lg(m_1) \rfloor$. Multiplying by the "page depth" gives the desired result.
\end{proof}

\end{frame}

\begin{frame}\frametitle{Memory Location Bound (Lemma 4.5.2)}
For any key $B_i$, if
\begin{align*}
k=&min_{j \in \{1, ..., h\}} \mid m'_j \geq location(B_i) \text{ then} \\
&k=min_{j \in \{1, ..., h\}} \mid m'_j \geq \frac{2}{p_i}-1.
\end{align*}
(I've omitted the result for gaps)
\begin{proof}
Give the depth of a node in $T$, we calculate its maximum BFS rank, which gives us its maximum memory level.
\end{proof}
\end{frame}

\begin{frame}\frametitle{Explicit Search Cost (Lemma 4.5.3)}
The cost of searching for key $B_i$, $C(B_i)$,can be bounded as follows: 
\begin{align*} 
C(B_i) \leq  \sum_{k'=1}^{k-1} &\left(\lfloor \lg(m'_{k'}+1) \rfloor - \lfloor \lg(m'_{k'-1}+1) \rfloor \right)\cdot c_{k'} + \\
 &\left(\lg(\frac{1}{p_i}) + 1 - \lfloor \lg(m'_{k-1}+1) \rfloor \right)\cdot c_k \\
&\text{such that } \left( k=min_{j \in \{1, ..., h\}} \mid m'_j \geq \frac{2}{p_i}-1 \right) 
\end{align*}
This, along with the bound for gaps is summed over all nodes to give a complete equation (Theorem 4.5.4).
\end{frame}

\begin{frame} \frametitle{Approximate Binary Search Trees of De Prisco and De Santis with Extensions by Bose and Dou\"{i}eb} \label{sec:deBST}

\begin{itemize}
\item[\textbf{Phase 1}] Create probability distribution with $2n$ zero, and $2n+1$ original probabilities. Use Bose and Dou\"{i}eb's BST algorithm to create an alphabetic tree.

\item[\textbf{Phase 2}] Move $p_i$ keys up \textit{starting tree} to the LCA of keys $q_{i-1}$ and $q_i$.

\item[\textbf{Phase 3}] Remove redundant edges.
\end{itemize}
\end{frame}

\begin{frame}\frametitle{BST Depth Bound (Lemma 4.6.1)}
We define $R$:
\begin{align*}
R = \{m' \text{ smallest access probabilities}\} \cup \{q_0, q_n\}
\end{align*}
\begin{align*} 
d_T(B_i) \leq
\begin{cases}  
\lfloor \lg(\frac{1}{p_i}) \rfloor - 1 &\text{ if } p_i \in R  \\
\lfloor \lg(\frac{1}{p_i}) \rfloor &\text{ otherwise.}
\end{cases} \\
d_T((B_{i-1},B_i)) \leq 
\begin{cases}  
\lfloor \lg(\frac{1}{q_i}) \rfloor &\text{ if } q_i \in R  \\
\lfloor \lg(\frac{1}{q_i}) \rfloor + 2 &\text{ if } q_i \text{ is } q_{max} \\
\lfloor \lg(\frac{1}{p_i}) \rfloor + 1 &\text{ otherwise.}
\end{cases}
\end{align*}


\begin{proof}
Result comes immediately from explanations within the work of Bose and Dou\"{i}eb \cite{bose2009efficient}.
\end{proof}

\end{frame}

\begin{frame}\frametitle{Memory Location Bound (Lemma 4.8.1)}
For any key $B_i$, if
\begin{align*}
k=&min_{j \in \{1, ..., h\}} \mid m'_j \geq location(B_i) \text{ then} \\
&k=min_{j \in \{1, ..., h\}} \mid m'_j \geq \frac{1}{p_i}-1 \text{ if } p_i \in R \\
&k=min_{j \in \{1, ..., h\}} \mid m'_j \geq \frac{2}{p_i}-1 \text{ otherwise.}
\end{align*}
(I've omitted the result for gaps)
\begin{proof}
Give the depth of a node in $T$, we calculate its maximum BFS rank, which gives us its maximum memory level.
\end{proof}
\end{frame}

\begin{frame}\frametitle{Explicit Search Cost (Lemma 4.8.2)}
\begin{align*} 
C(B_i) \leq 
\begin{cases}
 \sum_{k'=1}^{k-1} &\left(\lfloor \lg(m'_{k'}+1) \rfloor - \lfloor \lg(m'_{k'-1}+1) \rfloor \right)\cdot c_{k'}+ \\
 &\left(\lg(\frac{1}{p_i}) - \lfloor \lg(m'_{k-1}+1) \rfloor \right)\cdot c_k\\
&\text{such that } k=min_{j \in \{1, ..., h\}} \mid m'_j \geq \frac{1}{p_i}-1 \text{ if } p_i \in R \\\\
 \sum_{k'=1}^{k-1} &\left(\lfloor \lg(m'_{k'}+1) \rfloor - \lfloor \lg(m'_{k'-1}+1) \rfloor \right)\cdot c_{k'}+ \\
 &\left(\lg(\frac{1}{p_i}) + 1 - \lfloor \lg(m'_{k-1}+1) \rfloor \right)\cdot c_k\\
&\text{such that } k=min_{j \in \{1, ..., h\}} \mid m'_j \geq \frac{2}{p_i}-1 \text{ otherwise.}  
\end{cases}
\end{align*}
This, along with the bound for gaps is summed over all nodes to give a complete equation (Theorem 4.8.3).
\end{frame}

\begin{frame}

\begin{itemize}

\item Vaishnavi et al. \cite{vaishnavi1980optimum}, and Gotlieb  \cite{gotlieb1981optimal} independently solved optimally in $O(k\cdot n^3)$ time

\item In 1997, Becker propsed an $O(Dkn)$ time solution where D is the height of the tree \cite{becker1997construction}
\begin{itemize}
\item Thought, but not proven, to be close to optimal
\end{itemize}

\item Bose and Dou\"{i}eb's 2009 work also solved this problem with an $O(n)$ solution with worst case expected cost:
\begin{scriptsize}
\begin{align*}
\frac{H}{\lg(2k-1)} \leq C_{OPT} \leq \mathbf{C_{B \& D}} \leq \frac{H}{\lg k} + 1 + \sum_{i=0}^n q_i - q_0 - q_n - \sum_{i=0}^m q_{\text{rank}[i]}
\end{align*}
\end{scriptsize}

\end{itemize}


\end{frame}

\begin{frame}
\begin{itemize}
\item Formally, $R_1, R_2, ...$ unlimited number of registers each with location in memory.
\item Set of memories $M_1, M_2, ...M_l$
\item With memory sizes $m_1, m_2, ..., m_l$
\item Cost of accessing an item in $M_i$ is $c_i$
\item We assume $c_1 < c_2 < ... < c_l$
\item Cost of accessing item at location $a$:
\begin{align*}
\mu (a) = c_i \text{ if } \sum_{j = 1}^{i-1}m_j  < a \leq \sum_{j = 1}^{i}m_j.
\end{align*}
\end{itemize}
\end{frame}

\backupend
\end{document}