\documentclass[]{beamer}

\usepackage{beamerthemesplit} 
\usepackage{tikz}

\newcommand*\rfrac[2]{{#1}/{#2}}
\makeatletter
\newcommand{\rom}[1]{\romannumeral #1}

\tikzset{
    current/.style={
		scale=0.8,
		fill=blue,
		minimum size=0.5pt,
        draw,
        circle,
    }
}
\tikzset{
    select/.style={
		scale=0.8,
		fill=red,
		minimum size=0.5pt,
        draw,
        circle,
    }
}
\tikzset{
    max/.style={
		scale=0.8,
		minimum size=0.5pt,
        draw,
        circle,
    }
}

\title{Approximately Optimum Search
Trees in External Memory Models}
\subtitle{Master of Mathematics Presentation}
\author{Oliver Grant \\
Supervisor: Ian Munro}



\begin{document} \date{}
\begin{frame}
  \titlepage
\end{frame}

%\tableofcontents

\section{What is Prismata?}

%------------------------------------
\begin{frame}
\frametitle{What is Prismata?}

  \begin{itemize}
   \item A turn-based strategy game developed by Lunarch Studios.
   \item Founders are smart (MIT PhDs in CS/Math)
   \item Idea was to create a deep, challenging game that they would enjoy (started out as a hobby).
  \end{itemize}
  
\end{frame}

%------------------------------------

\begin{frame}
\frametitle{What Makes Prismata Interesting}
  \begin{itemize}
   \item Card set randomized each game. \textbf{Can't memorize openings!}
   \item This gives roughly ${80 \choose 8} \approx$ \textit{30 billion} different games.
   \item Card variety leads to interesting counters and combinations.
   \item The game is totally deterministic (other than randomized set and turn order). \textbf{No luck!}
   \item The game tree is massive. Complex turns could have several hundred possible moves.
  	
  \end{itemize}
  
\end{frame}

%------------------------------------

\begin{frame}
\frametitle{Prismata's Current AI}
\begin{center}
\cite{dave}
\end{center}
  \begin{itemize}
   \item Built by Dave Churchill, an AI PhD student from U of Alberta.
   \item Uses Alpha-Beta pruning with a small ply.
   \item Fast, reliable, beats beginners, but no match for good players. 	
  \end{itemize}
  
\end{frame}

%------------------------------------

\section{A Literature Review}

\begin{frame}
\frametitle{Some literature}
  \begin{itemize}
   \item Temporal difference learning and TD-Gammon. \cite{tesauro1995temporal}
   \item Reinforcement learning in board games.  \cite{ghory2004reinforcement}
   \item Temporal difference learning of position evaluation in the game of Go. \cite{schraudolph1994temporal}
   \item Several others...
   \item Provincial: A kingdom-adaptive AI for Dominion \cite{Fisher2014} 
  \end{itemize}
  
\end{frame}

%------------------------------------


\begin{frame}
\frametitle{Provincial by Matthew Fisher}
\begin{center}

 \begin{small}\begin{small}\begin{small}
 \url{http://graphics.stanford.edu/~mdfisher/DominionAI.html} \cite{Fisher2014}
 \end{small}\end{small}\end{small}
\end{center}

  \begin{itemize}
   \item Dominion AI built using Genetic Algorithms (GA).
   \item The set of cards chosen for each game is randomized.
   \item Players adapt purchase strategies based on card interactions.
   \item GA learns \textbf{buy} strategy, another AI takes care of \textbf{play} strategy.
   \item It takes a few minutes to learn a set, but is very strong.
  \end{itemize}
  
\end{frame}

%------------------------------------

\section{GA in Prismata}

\begin{frame}
\frametitle{My Plan for Prismata}
  \begin{itemize}
   \item Implement generational co-evolution (GCE) for \textbf{buy} strategies.
	\item Churchill's AI will take care of \textbf{play} strategies and end game. 
   \item GCE: Select dominant strategies, create larger pool of strategies, run a tournament, select best, mutate and repeat.
   \item A buy strategy will be an ordered-list of card-integer pairs: which unit and how many to buy.\\
   
\begin{center}
 \url{play.prismata.net} \cite{lunarch}
\end{center}



   
  \end{itemize}
  
\end{frame}

%------------------------------------

\section{Conclusion}

\begin{frame}
\frametitle{What I've done}
  \begin{itemize}
   \item Gone through several papers/other works.
   \item Have explored the source code from Lunarch Studios.
   \item Thoroughly examined the Provincial Dominion AI.
   \item Played \textbf{\textit{a lot}} of Prismata. (That counts right?)
  \end{itemize}

\begin{center}
 \url{play.prismata.net} \cite{lunarch}
\end{center}
  
\end{frame}

%------------------------------------


\begin{frame}
\frametitle{What's Next?}
  \begin{itemize}
   \item More reading! Grammatical Evolution [http://ncra.ucd.ie/Site/GEVA.html] 
   \item Multiagent Systems: Algorithmic, Game-Theoretic,
and Logical Foundations (Chapter 7.7 from this Book) \cite{shoham2008multiagent}
	\item Write some code.
	\item Testing: vs best AI and vs humans.
	\item Extensions? Monte Carlo Tree Search, Reinforcement Learning (CS 686 Project)
	\item Want a \textbf{Beta Key}? See me after class.

  \end{itemize}
  
\end{frame}

%------------------------------------


\bibliographystyle{abbrv}
\bibliography{ProjectPresentation}


\end{document}
