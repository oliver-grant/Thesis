 % uWaterloo Thesis Template for LaTeX 
% Last Updated May 24, 2011 by Stephen Carr, IST Client Services
% FOR ASSISTANCE, please send mail to rt-IST-CSmathsci@ist.uwaterloo.ca
% ADDED THIS LINE FROM THE OTHER SIDE
% Effective October 2006, the University of Waterloo 
% requires electronic thesis submission. See the uWaterloo thesis regulations at
% http://www.grad.uwaterloo.ca/Thesis_Regs/thesistofc.asp.

% DON'T FORGET TO ADD YOUR OWN NAME AND TITLE in the "hyperref" package
% configuration below. THIS INFORMATION GETS EMBEDDED IN THE PDF FINAL PDF DOCUMENT.
% You can view the information if you view Properties of the PDF document.

% Many faculties/departments also require one or more printed
% copies. This template attempts to satisfy both types of output. 
% It is based on the standard "book" document class which provides all necessary 
% sectioning structures and allows multi-part theses.

% DISCLAIMER
% To the best of our knowledge, this template satisfies the current uWaterloo requirements.
% However, it is your responsibility to assure that you have met all 
% requirements of the University and your particular department.
% Many thanks to the feedback from many graduates that assisted the development of this template.

% -----------------------------------------------------------------------

% By default, output is produced that is geared toward generating a PDF 
% version optimized for viewing on an electronic display, including 
% hyperlinks within the PDF.
 
% E.g. to process a thesis called "mythesis.tex" based on this template, run:

% pdflatex mythesis	-- first pass of the pdflatex processor
% bibtex mythesis	-- generates bibliography from .bib data file(s) 
% pdflatex mythesis	-- fixes cross-references, bibliographic references, etc
% pdflatex mythesis	-- fixes cross-references, bibliographic references, etc

% If you use the recommended LaTeX editor, Texmaker, you would open the mythesis.tex
% file, then click the pdflatex button. Then run BibTeX (under the Tools menu).
% Then click the pdflatex button two more times. If you have an index as well,
% you'll need to run MakeIndex from the Tools menu as well, before running pdflatex
% the last two times.

% N.B. The "pdftex" program allows graphics in the following formats to be
% included with the "\includegraphics" command: PNG, PDF, JPEG, TIFF
% Tip 1: Generate your figures and photos in the size you want them to appear
% in your thesis, rather than scaling them with \includegraphics options.
% Tip 2: Any drawings you do should be in scalable vector graphic formats:
% SVG, PNG, WMF, EPS and then converted to PNG or PDF, so they are scalable in
% the final PDF as well.
% Tip 3: Photographs should be cropped and compressed so as not to be too large.

% To create a PDF output that is optimized for double-sided printing: 
%
% 1) comment-out the \documentclass statement in the preamble below, and
% un-comment the second \documentclass line.
%
% 2) change the value assigned below to the boolean variable
% "PrintVersion" from "false" to "true".

% --------------------- Start of Document Preamble -----------------------

% Specify the document class, default style attributes, and page dimensions
% For hyperlinked PDF, suitable for viewing on a computer, use this:
\documentclass[letterpaper,12pt,titlepage,oneside,final]{book}
 
% For PDF, suitable for double-sided printing, change the PrintVersion variable below
% to "true" and use this \documentclass line instead of the one above:
%\documentclass[letterpaper,12pt,titlepage,openright,twoside,final]{book}

% Some LaTeX commands I define for my own nomenclature.
% If you have to, it's better to change nomenclature once here than in a 
% million places throughout your thesis!
\newcommand{\package}[1]{\textbf{#1}} % package names in bold text
\newcommand{\cmmd}[1]{\textbackslash\texttt{#1}} % command name in tt font 
\newcommand{\href}[1]{#1} % does nothing, but defines the command so the
    % print-optimized version will ignore \href tags (redefined by hyperref pkg).
%\newcommand{\texorpdfstring}[2]{#1} % does nothing, but defines the command
% Anything defined here may be redefined by packages added below...

\usepackage{algorithm}
\usepackage{algpseudocode}
\usepackage{pifont}
\usepackage{amsmath}
\usepackage{amssymb}
\usepackage[pdftex]{graphicx}
\usepackage{enumerate}
\usepackage{amsfonts}
\usepackage{url}
\usepackage{amsmath,amsfonts,amssymb,amsthm,epsfig,epstopdf,titling,url,array}
\usepackage[]{units}
\usepackage{xcolor}
\theoremstyle{plain}
\newtheorem{thm}{Theorem}[section]
\newtheorem{lem}[thm]{Lemma}
\newtheorem{prop}[thm]{Proposition}
\newtheorem*{cor}{Corollary}


% This package allows if-then-else control structures.
\usepackage{ifthen}
\newboolean{PrintVersion}
\setboolean{PrintVersion}{false} 
% CHANGE THIS VALUE TO "true" as necessary, to improve printed results for hard copies
% by overriding some options of the hyperref package below.

%\usepackage{nomencl} % For a nomenclature (optional; available from ctan.org)
\usepackage{amsmath,amssymb,amstext} % Lots of math symbols and environments
\usepackage[pdftex]{graphicx} % For including graphics N.B. pdftex graphics driver 

% Hyperlinks make it very easy to navigate an electronic document.
% In addition, this is where you should specify the thesis title
% and author as they appear in the properties of the PDF document.
% Use the "hyperref" package 
% N.B. HYPERREF MUST BE THE LAST PACKAGE LOADED; ADD ADDITIONAL PKGS ABOVE
\usepackage[pdftex,letterpaper=true,pagebackref=false]{hyperref} % with basic options
		% N.B. pagebackref=true provides links back from the References to the body text. This can cause trouble for printing.
\hypersetup{
    plainpages=false,       % needed if Roman numbers in frontpages
    pdfpagelabels=true,     % adds page number as label in Acrobat's page count
    bookmarks=true,         % show bookmarks bar?
    unicode=false,          % non-Latin characters in Acrobat’s bookmarks
    pdftoolbar=true,        % show Acrobat’s toolbar?
    pdfmenubar=true,        % show Acrobat’s menu?
    pdffitwindow=false,     % window fit to page when opened
    pdfstartview={FitH},    % fits the width of the page to the window
    pdftitle={uWaterloo\ LaTeX\ Thesis\ Template},    % title: CHANGE THIS TEXT!
%    pdfauthor={Author},    % author: CHANGE THIS TEXT! and uncomment this line
%    pdfsubject={Subject},  % subject: CHANGE THIS TEXT! and uncomment this line
%    pdfkeywords={keyword1} {key2} {key3}, % list of keywords, and uncomment this line if desired
    pdfnewwindow=true,      % links in new window
    colorlinks=true,        % false: boxed links; true: colored links
    linkcolor=blue,         % color of internal links
    citecolor=green,        % color of links to bibliography
    filecolor=magenta,      % color of file links
    urlcolor=cyan           % color of external links
}
\ifthenelse{\boolean{PrintVersion}}{   % for improved print quality, change some hyperref options
\hypersetup{	% override some previously defined hyperref options
%    colorlinks,%
    citecolor=black,%
    filecolor=black,%
    linkcolor=black,%
    urlcolor=black}
}{} % end of ifthenelse (no else)

% Setting up the page margins...
% uWaterloo thesis requirements specify a minimum of 1 inch (72pt) margin at the
% top, bottom, and outside page edges and a 1.125 in. (81pt) gutter
% margin (on binding side). While this is not an issue for electronic
% viewing, a PDF may be printed, and so we have the same page layout for
% both printed and electronic versions, we leave the gutter margin in.
% Set margins to minimum permitted by uWaterloo thesis regulations:
\setlength{\marginparwidth}{0pt} % width of margin notes
% N.B. If margin notes are used, you must adjust \textwidth, \marginparwidth
% and \marginparsep so that the space left between the margin notes and page
% edge is less than 15 mm (0.6 in.)
\setlength{\marginparsep}{0pt} % width of space between body text and margin notes
\setlength{\evensidemargin}{0.125in} % Adds 1/8 in. to binding side of all 
% even-numbered pages when the "twoside" printing option is selected
\setlength{\oddsidemargin}{0.125in} % Adds 1/8 in. to the left of all pages
% when "oneside" printing is selected, and to the left of all odd-numbered
% pages when "twoside" printing is selected
\setlength{\textwidth}{6.375in} % assuming US letter paper (8.5 in. x 11 in.) and 
% side margins as above
\raggedbottom

% The following statement specifies the amount of space between
% paragraphs. Other reasonable specifications are \bigskipamount and \smallskipamount.
\setlength{\parskip}{\medskipamount}

% The following statement controls the line spacing.  The default
% spacing corresponds to good typographic conventions and only slight
% changes (e.g., perhaps "1.2"), if any, should be made.
\renewcommand{\baselinestretch}{1} % this is the default line space setting

% By default, each chapter will start on a recto (right-hand side)
% page.  We also force each section of the front pages to start on 
% a recto page by inserting \cleardoublepage commands.
% In many cases, this will require that the verso page be
% blank and, while it should be counted, a page number should not be
% printed.  The following statements ensure a page number is not
% printed on an otherwise blank verso page.
\let\origdoublepage\cleardoublepage
\newcommand{\clearemptydoublepage}{%
  \clearpage{\pagestyle{empty}\origdoublepage}}
\let\cleardoublepage\clearemptydoublepage

%======================================================================
%   L O G I C A L    D O C U M E N T -- the content of your thesis
%======================================================================
\begin{document}

% For a large document, it is a good idea to divide your thesis
% into several files, each one containing one chapter.
% To illustrate this idea, the "front pages" (i.e., title page,
% declaration, borrowers' page, abstract, acknowledgements,
% dedication, table of contents, list of tables, list of figures,
% nomenclature) are contained within the file "uw-ethesis-frontpgs.tex" which is
% included into the document by the following statement.
%----------------------------------------------------------------------
% FRONT MATERIAL
%----------------------------------------------------------------------
% T I T L E   P A G E
% -------------------
% Last updated May 24, 2011, by Stephen Carr, IST-Client Services
% The title page is counted as page `i' but we need to suppress the
% page number.  We also don't want any headers or footers.
\pagestyle{empty}
\pagenumbering{roman}

% The contents of the title page are specified in the "titlepage"
% environment.
\begin{titlepage}
        \begin{center}
        \vspace*{1.0cm}

        \Huge
        {\bf Approximately Optimum Search Trees in External Memory Models }

        \vspace*{1.0cm}

        \normalsize
        by \\

        \vspace*{1.0cm}

        \Large
        Oliver Grant \\

        \vspace*{3.0cm}

        \normalsize
        A thesis \\
        presented to the University of Waterloo \\ 
        in fulfillment of the \\
        thesis requirement for the degree of \\
        Master of Mathematics \\
        in \\
        Computer Science \\

        \vspace*{2.0cm}

        Waterloo, Ontario, Canada, 2016 \\

        \vspace*{1.0cm}

        \end{center}
\end{titlepage}

% The rest of the front pages should contain no headers and be numbered using Roman numerals starting with `ii'
\pagestyle{plain}
\setcounter{page}{2}

\cleardoublepage % Ends the current page and causes all figures and tables that have so far appeared in the input to be printed.
% In a two-sided printing style, it also makes the next page a right-hand (odd-numbered) page, producing a blank page if necessary.
 


% D E C L A R A T I O N   P A G E
% -------------------------------
  % The following is the sample Delaration Page as provided by the GSO
  % December 13th, 2006.  It is designed for an electronic thesis.
  \noindent
I hereby declare that I am the sole author of this thesis. This is a true copy of the thesis, including any required final revisions, as accepted by my examiners.

  \bigskip
  
  \noindent
I understand that my thesis may be made electronically available to the public.

\cleardoublepage
%\newpage

% A B S T R A C T
% ---------------

\begin{center}\textbf{Abstract}\end{center}

We examine optimal and near optimal solutions to the classic binary search tree problem of Knuth \cite{knuth1971optimum}. First, we re-examine a solution of G{\"u}ttler, Melhorn and Schneider \cite{guttler1980binary} which was shown to have a worst case bound of $c \cdot H + 2$ where $c \geq \frac{1}{H(\frac{1}{3},\frac{2}{3})} \approx 1.08$. We give an improved worst case bound on the heuristic of $H+4$. Next, we examine the optimum BST problem under a model of external memory. We use the Hierarchical Memory Model (HMM) of Aggarwal et al. \cite{aggarwal1987model} and propose two approximate solutions which run in $O(n)$ time where $n$ is the number of words in our data set. Using these methods, we improve upon a bound given in Thite's 2001 thesis under the related HMM$_2$ model in the approximate setting. We also examine related problem of binary trees on multisets of probabilities. We provide a simply $O(n \lg(n))$ algorithm that is within $1$ of optimal.

\cleardoublepage
%\newpage

% A C K N O W L E D G E M E N T S
% -------------------------------

\begin{center}\textbf{Acknowledgements}\end{center}

I would like to thank all the little people who made this possible. TODO
\cleardoublepage
%\newpage

% D E D I C A T I O N
% -------------------

\begin{center}\textbf{Dedication}\end{center}

This is dedicated to my partner Melissa.
\cleardoublepage
%\newpage

% T A B L E   O F   C O N T E N T S
% ---------------------------------
\renewcommand\contentsname{Table of Contents}
\tableofcontents
\cleardoublepage
\phantomsection
%\newpage

% L I S T   O F   T A B L E S
% ---------------------------
\addcontentsline{toc}{chapter}{List of Tables}
\listoftables
\cleardoublepage
\phantomsection		% allows hyperref to link to the correct page

% L I S T   O F   F I G U R E S
% -----------------------------
\addcontentsline{toc}{chapter}{List of Figures}
\listoffigures
\cleardoublepage
\phantomsection		% allows hyperref to link to the correct page
%\newpage

% L I S T   O F   S Y M B O L S
% -----------------------------
% To include a Nomenclature section
% \addcontentsline{toc}{chapter}{\textbf{Nomenclature}}
% \renewcommand{\nomname}{Nomenclature}
% \printglossary
% \cleardoublepage
% \phantomsection % allows hyperref to link to the correct page
% \newpage

% Change page numbering back to Arabic numerals
\pagenumbering{arabic}

 

%----------------------------------------------------------------------
% MAIN BODY
%----------------------------------------------------------------------
% Because this is a short document, and to reduce the number of files
% needed for this template, the chapters are not separate
% documents as suggested above, but you get the idea. If they were
% separate documents, they would each start with the \chapter command, i.e, 
% do not contain \documentclass or \begin{document} and \end{document} commands.

\chapter{Introduction}


\section{Binary Search Trees}

A binary search tree is simple structure used to store key-value pairs. It was invented in the late 1950s and early 1960s and is generally attributed to the combined efforts of Windley, Booth, Colin and Hibbard \cite{windley1960trees} \cite{booth1960efficiency} \cite{hibbard1962some}. In general, the trees allow for quick binary searches through the data in search of a specific key. Each node has a key over which there is a total-ordering (a number, a string, etc.), as well as some value (generally the important information). Each tree node has at most two children generally labelled as the \textit{left} and \textit{right}. All nodes in the subtree of the \textit{left} child of a specific node have a key strictly less than the key of the node in question. Similarly, nodes in the subtree of the \textit{right} child of a specific node have a key strictly greater than the key of the node in question. A pointer is typically stored to a root node. Search begins from this root node and is done by recursively searching in either the \textit{left} or \textit{right} child of root node, or stopping if the root being searched has the correct key.

\section{The Optimum Binary Search Tree Problem}

Knuth first proposed the optimum binary search tree problem in 1971 \cite{knuth1971optimum}. We are given a set of $n$ words $B_1, B_2, ..., B_n$ and $2n+1$ frequencies, ${p_1, p_2, ..., p_n}$, ${q_0, q_1, ..., q_n}$ representing the probabilities of searching for each given word and the probabilities of searching for strings between (and outside of) these words. We have that $ q_0 + \sum\limits_{i=1}^n p_i+q_i = 1$. We also assume that without loss of generality $q_i+p_i+q_{i+1} \neq 0$ for any $i$. The words (and gaps between) are used as keys and the lexicographical ordering of them provides our order over the keys. Our goal is to construct a binary tree such that the expected cost of search is minimized. The names make up the leaves of the tree while, gaps make up the internal nodes. The weighted path length of the tree is: \\
$P = \sum_{i=1}^{n} p_i(b_i+1) + \sum_{j=1}^{n} q_j(a_j)$ \\
Where $b_i$ and $a_j$ represent the depth of nodes representing the $i^{th}$ word and $j^{th}$ gap respectively. The optimal solution of Knuth requires $O(n^2)$ time, and $O(n^2)$ space. This solution is both time and space intensive. We will later examine an approximate solution  to this problem of G{\"u}ttler, Mehlhorn and Schneider which uses $O(n^2)$ time but $O(n)$ space and improve its worst-case bound. However, these problems were examined under the RAM model which is an inadequate model for many situations. We examine the problem in more realistic models and look at approximate solutions under these settings.

\section{Why Study Binary Search Trees}

Binary search trees are ubiquitous throughout computer science with numerous applications. The basic binary search tree has been built upon in many ways. AVL trees (named after creators AdelsonVelskii and Landis) were the first form of self-balancing binary search trees introduced \cite{adelsonvelskii1963algorithm}. The trees invented by the pair in 1963 and maintain a height of $O(lg(n))$ where $n$ is the number of nodes in the tree during insertions and deletions (both of which take $O(lg(n))$ time). Improved self-balancing binary search trees followed in the form of red-black trees by Bayer in 1972 and splay trees by Sleator and Tarjan in 1985 \cite{bayer1972symmetric} \cite{sleator1985self}. Tango trees were invented in 2007 by Demaine et al. and provided the first $O(lg lg n)$ competitive binary tree \cite{demaine2007dynamic}. Here, $O(lg lg n)$ competitive means that the tango tree does at most $O(lg lg n)$ times more work (pointer movements and rotations) than an optimal offline tree. B trees are among the most commonly used binary tree variant and were invented in 1970 by Bayer and McCeight \cite{bayer1970organization}. 


Binary space partition
Binary Tries in routers
Hash Trees
Heaps - PQ
Huffman Code - compression
Syntax tree
expression evaluation

\section{Overview}

In Chapter 2 I review previous work done in the areas of binary search trees, multiway trees, alphabetic trees and various models of external memory. In Chapter 3, I re-examine the Modified Entropy Rule (ME) of G{\"u}ttler, Mehlhorn and Schneider \cite{guttler1980binary}. This is a $O(n)$ time algorithm for approximating the optimum binary search tree problem in the RAM model. The method worked very well in practice, and the group had great experimental results, but unfortunately they could not bound the worst case expected cost of the as they would have hoped. While simpler solutions like the \textit{Min-max} and \textit{Weight Balanced} techniques of Bayer \cite{bayer1975improved} had worst case costs of at most $H+2$, the trio's ME technique had a worst case expected search cost of $c*H+2$ where $c \approx 1.08$. I provide a new argument of its worst case expected search cost and show that it is within a constant of entropy: at worst $H+4$. In Chapter 4, I move on to external memory models, examining the optimum binary search tree problem under the hierarchical memory model as was done in Thite's thesis \cite{thite2008optimum}. I provide TWO? $O(n*lg(m_1))$ time algorithm which has a worst case expected cost of TODO FIX $C \leq (\lceil lg(m_1) \rceil * (\lfloor log_{m_1}(\frac{2}{min_{p,q}}) \rfloor + 1)) * W * (H + 1)$ WHERE TODO. The solution provided also gives a direct improvement over the solution Thite provided in the same work for HMM$_2$. In Chapter 5, I extend my solutions to examine a more realistic model for external memory, the hierarchical memory with block transfers model (HMBTM). I use similar algorithms which run in TODO and give worst case expected search costs of TODO. In Chapter 6 I summarize my findings and discuss several problems which remain open.


\chapter{Background and Related Work}



\section{Binary Search Trees}

After Knuth's initial examination of the optimum binary search tree problem in 1971 \cite{knuth1971optimum}, several others have examined the approximate version of the problem. Knuth's optimal solution requires $O(n^2)$ time and space which is too costly in many situations. While unable to bound an approximate algorithm within a constant of the optimal solution, many authors have been able to bound the cost based on the entropy of the distribution of probabilities, $H$. Specifically, $H = \sum_{i=1}^{n} p_i*log_2(\frac{1}{p_i}) + \sum_{j=0}^{n} q_i*lg(\frac{1}{q_j})$. In 1975, Bayer showed that $H-lg H-(lg e-1) \leq C_{Opt}$, $C_{Opt} \leq C_{WB} \leq H + 2$ and $C_{Opt} \leq C_{MM} \leq H + 2$ where $C_{Opt}, C_{WB}$, and $C_{MM}$ are costs for the optimal solution, as well as weight-balanced and min-max heuristic methods respectively \cite{bayer1975improved}. Weight-balanced and min-max costs heuristics are greedy and require both $O(n)$ time and $O(n)$ space to run. In 1980, G{\"u}ttler, Mehlhorn and Schneider presented a new heuristic, the Modified Entropy Rule (ME) \cite{guttler1980binary} which built upon the ideas of Horibe \cite{horibe1977improved}. G{\"u}ttler, Mehlhorn AND Schneider gave empirical evidence that the heuristic out-performed others \cite{guttler1980binary}. While the heuristic took $O(n^2)$ time, it only required $O(n)$ space, a huge savings over the optimal solution. However, they were unable to prove that $C_{ME} \leq H+2$ (like previous weight-balanced and min-max heuristics) and settled with $C_{ME} \leq c_1*H+2$ where $c_1=\frac{1}{H(\frac{1}{3}, \frac{2}{3})} \approx 1.08$. In 1993, De Prisco and De Santis presented a new heuristic for constructing a near-optimum binary search tree \cite{de1993binary}. The method is discussed in more detail in section~\ref{sec:deBST} and has an upper bounded cost of at most $H+1-q_0-q_n+q_{max}$ where $q_{max}$ is the maximum weight leaf node. This method was later updated by Bose and Dou\"{i}eb to have a worst case cost of 
 \begin{center}
$\frac{H}{lg(2k-1)} \leq P_{OPT} \leq P_T \leq \frac{H}{lg k} + 1 + \sum_{i=0}^n q_i - q_0 - q_n - \sum_{i=0}^m q_{rank[i]}$ \cite{bose2009efficient}
\end{center}
Here, H is the entropy of the probability distribution, $P_{OPT}$ is the average path-length in the optimal tree, $P_T$ is the average path length of the tree built using their algorithm and $m=max({n-3P,P})-1 \geq \frac{n}{4} - 1$. $P$ is the number of increasing or decreasing sequences in a left-to-right read of the access probabilities of the leaves. Moreover, $q_{rank[i]}$ is the $i^{th}$ smallest access probability among all leaves except $q_0$ and $q_n$. \\ 
 
\section{Alphabetic Trees}

Given a set of $n$ keys with various probabilities, the problem is to build a binary search tree where every internal node has two children, leaves have no children, and the $n$ keys described are the leaves with minimum expected search cost. Here, the expected search cost is $\sum p_i * l_i*$ where $p_i$ is the probability of searching for leaf/key $i$ and $l_i$ is the leaf's level in the tree. The alphabetic ordering of the leaves must be maintained. This is the same as the binary search tree problem with all internal node weights zero.

In 1952, Huffman famously developed the Huffman tree, which solved the same problem without a left-to-right ordering constraint on leaves \cite{huffman1952method}. Gilbert and Moore examined the problem with the added alphabetic constraint and developed a $O(n^3)$ algorithm which solved the problem optimally \cite{gilbert1959variable}. Hu and Tucker gave a $O(n^2)$ time and space algorithm in 1971 \cite{hu1971optimal} which was improved by Knuth to take only $O(n lg n)$ time and $O(n)$ space in 1973 \cite{knuth1973sorting}. The original proof of Hu and Tucker was extremely complicated, but was fortunately later simplified by Hu \cite{hu1973new} and Hu et al. \cite{hu1979binary}. Garsia and Wachs gave an independent $O(n lg n)$ time, $O(n)$ space algorithm in 1977 \cite{garsia1977new}. This was shown to be equivalent to the Hu and Tucker algorithm in 1982 by Hu \cite{Hu1982Book} and also went through a proof simplification \cite{kingston1988new} by Kingston in 1988.

In 1991, Yeung proposed an approximate solution which solved the problem in $O(n)$ time and space \cite{yeung1991alphabetic}. The algorithm produced a tree with worst case cost $H + 2 - q_1-q_n$. This was later imporved by De Prisco and De Santis who created an $O(n)$ time algorithm which had a worst case cost of $H+1-q_0-q_n+q_{max}$ \cite{de1993binary}. The method was improved one more time by Bose and Dou\"{i}eb who improved upon Yeung's method by decreasing the bound by $\sum_{i=0}^m q_{rank[i]}$ where $m=max({n-3P,P})-1 \geq \frac{n}{4} - 1$, $P$ is the number of increasing or decreasing sequences in a left-to-right read of the access probabilities of the leaves and $q_{rank[i]}$ is the $i^{th}$ smallest access probability among all leaves except $q_0$ and $q_n$ \cite{bose2009efficient}. De Prisco and De Santis used Yeung's method within their own, so this improvement to Yeung's method gave an overall tighter bound of $H+1-q_0-q_n+q_{max}-\sum_{i=0}^m q_{rank[i]}$. CHECK THIS OUT TODO

\section{Multiway Trees} 

This static k-ary or multiway search tree problem is similar to optimum binary search tree problem with the added constraint that up to $k$ keys can be placed into a single node, and cost of search within a node is constant. Multiway search trees maintain an ordering property similar to that of traditional binary search trees. Each key in every page in the subtree rooted at a specific location in a page $g$ (i.e. between two keys $k$ and $l$) must have its key lie between $k$ and $l$. Each internal node of the k-ary tree contains at least one and at most $k-1$ keys while a leaf node contains no keys. Successful searches end in an internal while unsuccessful searches end in one of the $n+1$ leaves of the tree. The cost of search is the average path depth is defined as :

\begin{center}
$\sum_{i=1}^{n} p_i(d_T(x_i)+1) + \sum_{j=0}^{n} q_j(d_T(x_{i-1},x_i))$
\end{center}

where $x_i$'s represent successful search keys, pairs $(x_{i-1},x_i)$ represent unsuccessful search "keys" and $d_T(x_i)$  or $d_T(x_{i-1},x_i)$ represent the depth of a specific successful or unsuccessful search respectively.

Vishnavi et al. \cite{vaishnavi1980optimum}, and Gotlieb  \cite{gotlieb1981optimal} in 1980 and 1981 respectively independently solved the problem optimally in $O(k*n^3)$ time. In a slightly modified B-tree model (every leaf has same depth, every internal node is at least half full), Becker's 1994 work gave a $O(kn^{\alpha})$ time algorithm where $\alpha=2+log_k 2$ \cite{becker1994new}. Moreover, in 1997 Becker propsed an $O(Dkn)$ time algorithm where D is the height of the resulting tree\cite{becker1997construction}. The algorithm did not produce an optimal tree but was thought to be empirically close despite having no strong upper bound. In 2009, Bose and Dou\"{i}eb gave both an upper and lower bound on the optimal search tree in terms of the entropy of the probability distribution as well as an $O(n)$ time algorithm to build a near-optimal tree \cite{bose2009efficient}. There bounds of:
\begin{center}
$\frac{H}{lg(2k-1)} \leq P_{OPT} \leq P_T \leq \frac{H}{lg k} + 1 + \sum_{i=0}^n q_i - q_0 - q_n - \sum_{i=0}^m q_{rank[i]}$
\end{center}
will be discussed in more detail in section BLAH TODO of this paper.

\section{Memory Models}

HMM
HMM BT
Other Stuff
Cache Oblivious
Look at what Thite wrote about
TODO - LOOK @ SURVEY ON MEMORY MODELS


\chapter{An Improved Bound for the Modified Minimum Entropy Heuristic}

\section{Preliminaries}

Recall that $H = \sum_{i=1}^{n} p_i*lg(\frac{1}{p_i}) + \sum_{j=0}^{n} q_i*lg(\frac{1}{q_j})$. We also use $H(x_1,x_2,...,x_n)$ to describe the entropy of probability distribution $(x_1, x_2, ...., x_n)$. For sub tree $t$, we let $p_t$ be its total probability. $P_{L}(p_i)$ and $P_{R}(p_i)$ are probabilities of searching for a word lexicographically before or after (respectively) word $B_i$. $P_{L_t}(p_i)=\frac{P_L(p_i)}{p_t}$ and $P_{R_t}(p_i)=\frac{P_R(p_i)}{p_t}$ are these same probabilities normalized for sub tree $t$. Similarly, $P_{L_t}(q_j)$ and $P_{R_t}(q_j)$ describe the normalized probabilities of searching for a word before or after (respectively) the lexicographic gap between words $B_j$ and $B_{j+1}$. We let $E_t=H(P_{L_t}(p_i), \frac{p_i}{p_t}, P_{R_t}(p_i))$ be the local entropy of a sub tree $t$ rooted at word $B_i$.  

\section{The Modified Minimum Entropy Heuristic}

As described in \cite{guttler1980binary}, the heuristic greedily chooses the word $B_i$ as the root such that $H(P_{L_t}(p_i), \frac{p_i}{p_t}, P_{R_t}(p_i))$ is maximized. There are two exceptions to this rule. First, if there exists $p_i$ such that $\frac{p_i}{p_t} > max(P_{L_t}(p_i), P_{R_t}(p_i))$ we always select $B_i$ as the root. Moreover, if there exists $q_j$ such that $\frac{q_j}{p_t} > max(P_{L_t}(q_j), P_{R_t}(q_j))$ then we select the root from among $B_j$ and $B_{j+1}$. $B_j$ is chosen if $P_{L_t}(q_j) > P_{R_t}(q_j)$ and $B_{j+1}$ is chosen otherwise. Since it takes linear time and constant space to find the all of the optimal local entropy splits in each level of the tree, the algorithm takes $O(n^2)$ time and $O(n)$ space.  \\

\section{ME is within 4 of Entropy}

\begin{lem}
When using the ME Rule to choose the root $B_r$ of a binary search tree $t$ with total probability $p_t$, one of the following three cases must occur: \\
1) $E_t \geq 1-2 \frac{p_r}{p_t}$ \\
2) There exists $q_i$ such that $\frac{q_i}{p_t} > max(P_{L_t}(q_i), P_{R_t}(q_i))$\\
3)  $max(P_{L_t}(p_r), P_{R_t}(p_r)) < \frac{4}{5} p_t$\\
\end{lem}
\begin{proof}
Suppose there exists some $p_i$ such that $\frac{p_i}{p_t} > max(P_{L_t}(p_i), P_{R_t}(p_i))$. By the ME Rule, it must be selected as the root and thus $p_r=p_i$. Moreover, both $P_{L_t}(p_i)$ and $P_{R_t}(p_i))$ must be less than one half. As shown in \cite{gallager1968information} $H(x,1-x) \geq 2x$ when $x<\frac{1}{2}$. Thus we have: \\
$E_t \geq H( max(P_{L_t}(p_i), P_{R_t}(p_i)), 1-max(P_{L_t}(p_i), P_{R_t}(p_i))$ \\
$ \geq 2*max(P_{L_t}(p_i), P_{R_t}(p_i))$ \\ $ \geq 1-\frac{p_i}{p_t}$ \\ 
$ \geq 1-2 \frac{p_i}{p_t} \geq 1-2 \frac{p_r}{p_t}$ \\
 as required. \\
 
 If we do not have some $p_i$ such that $\frac{p_i}{p_t} > max(P_{L_t}(p_i), P_{R_t}(p_i))$ but do have some $p_i$ such that $P_{L_t}(p_i) \leq \frac{1}{2}p_t$ and $P_{R_t}(p_i) \leq \frac{1}{2}p_t$) then we have: \\
 $E_t \geq H(P_{L_t}(p_i), \frac{p_i}{p_t} , P_{R_t}(p_i)$ where $0 \leq P_{L_t}(p_i) \leq 0.5$, $0 \leq \frac{p_i}{p_t} \leq 0.5$ and $0 \leq P_{R_t}(p_i) \leq 0.5$  and we know that \\
 $H(P_{L_t}(p_i), \frac{p_i}{p_t} , P_{R_t}(p_i)) \geq H(1/2, 1/2) = 1$ in this case.
 Thus, $E_t \geq 1-2p_r$ as required. \\
 
 Otherwise, we must have some $q_i$ spanning the middle of the data set (i.e. $P_{L_t}(q_i) < \frac{1}{2}p_t$ and $P_{R_t}(q_i) < \frac{1}{2}p_t$). Suppose that case 2) does not occur: there does not exist a $q_i$ such that $\frac{q_i}{p_t} > max(P_{L_t}(q_i), P_{R_t}(q_i))$. Then, we have that \\ $max(P_{L_t}(p_i), P_{R_t}(p_i)) \geq min(P_{L_t}(p_i), P_{R_t}(p_i))$ and \\ $max(P_{L_t}(p_i), P_{R_t}(p_i)) \geq q_i$ \\ Thus, $max(P_{L_t}(p_i), P_{R_t}(p_i)) \geq 1/3$ and by our assumption $max(P_{L_t}(p_i), P_{R_t}(p_i)) < \frac{1}{2}$. \\
 So, as in the proof of table 3 (5.3) in \cite{guttler1980binary}\\
 $E_t \geq H(1/3, 2/3) \approx 0.92$. \\

Now also suppose that case 1) does not occur. 
$\implies E_t < 1-2\frac{p_r}{q_t}$ \\
$\implies \frac{p_r}{q_t} < \frac{1-H(\frac{1}{3}, \frac{2}{3})}{2} \approx 0.04$ \\
Suppose that $max(P_{L_t}(p_r), P_{R_t}(p_r)) \geq \frac{4}{5} p_t$ then we have \\
$E_t \leq H(\frac{4}{5}, \frac{1-H(\frac{1}{3}, \frac{2}{3})}{2}, \frac{1}{5}-\frac{1-H(\frac{1}{3}, \frac{2}{3})}{2}) \approx 0.87 < H(\frac{1}{3}, \frac{2}{3}) \geq E_t$ \\ 
a contradiction. \\
Thus, if we do not have case 1) or 2), we must have case 3). Since this is our last remaining case, this completes the proof.
 

\end{proof}


\begin{thm}
$C_{ME} \leq H + 4$
\end{thm}

\begin{proof}
This uses a similar style to the proof of theorem 4.4 in \cite{bayer1975improved}.
We will bind each $E_t$ for each sub tree of our BST on a case by case basis using the cases of \textbf{Lemma 3.3.1}.\\
If case 1 occurs, we obviously have that \\
$E_t \geq 1-2 \frac{p_r}{p_t}$ \\
Note that this can only happen once for each word (a word can only be root once). \\

While it wasn't explicitly stated, in \textbf{Lemma 3.3.1} it was shown that if some $p_i$ spans in the middle of the data set, $P_{L_t}(p_i) \leq \frac{1}{2}p_t$ and $P_{R_t}(p_i) \leq \frac{1}{2}p_t$, then regardless of whether or not $\frac{p_i}{p_t} > max(P_{L_t}(p_i), P_{R_t}(p_i))$, we can still show that case 1 occurs. Suppose for the remainder of the proof that there is no such middle-spanning $p_i$.

Let $q_m$ be the unique middle gap (i.e. $P_{L_t}(q_m) < \frac{1}{2} p_t$ and $P_{R_t}(q_m) < \frac{1}{2} p_t$) when case 2 or 3 occurs. When case 2 occurs we have that: \\
$E_t \geq H(\frac{1}{2}-\frac{1}{2} \frac{q_m}{p_t}, \frac{1}{2} + \frac{1}{2} \frac{q_m}{p_t}) \geq 2(\frac{1}{2}-\frac{1}{2} \frac{q_m}{p_t})=1-\frac{q_m}{p_t}$. \\
Note that by the definition of the ME Rule, when this occurs, $q_m$ must be a leaf of depth at most 2. Thus, this condition can only happen twice for each $q_m$. \\

When case 2 does not occur (and we have a $q_m$ spanning the middle) we must have case 3. This gives us: \\
$E_t \geq H(\frac{1}{2}-\frac{1}{2} \frac{q_m}{p_t}, \frac{1}{2} + \frac{1}{2} \frac{q_m}{p_t}) \geq 1- \frac{4}{5} (\frac{q_m}{p_t})^2$ when $\frac{q_m}{p_t} < \frac{1}{2}$ (which must be true since case 2 did not occur). \\

\iffalse
TODO PROVE THIS DERIVATIVE
derivative of (0.5-0.5*q)*log_2(1/(0.5-0.5*q)) + (0.5+0.5*q)*log_2(1/(0.5+0.5*q)) - (1-(4/5)*(q^2))
\fi

As in \cite{bayer1975improved} we define a $b_t$ for each sub tree $t$ as follows: \\
let $b_t=2p_r$ for case 1. \\
let $b_t=2q_m$ for case 2. \\
let $b_t=\frac{q_m^2}{p_t}$ for case 3. \\
Note that, in 1975 Bayer showed that: \\
$C = \sum_{t \in S_T} p_t$ and $H = \sum_{t \in S_T} P_t E_t $ Where $S_T$ is the set of all sub trees of our tree (\textbf{Lemma 2.3} \cite{bayer1975improved}).
Thus, we have\\
$H = \sum_{t \in S_T} P_t E_t \geq \sum_{t \in S_T} P_t - \sum b_t = C - \sum b_t$ \\
$ \implies C \leq H + \sum b_t$ \\

As mentioned above, cases 1 and 2 can only occur once and twice respectively. Case 3 however, can occur many times, but each time it occurs, $\frac{q_m}{p_t}$ must decrease by a factor of $\frac{5}{4}$ since $max(P_{L_t}(p_r), P_{R_t}(p_r)) < \frac{4}{5} p_t$ (we recursively examine a smaller sub tree). Moreover, if $\frac{q_m}{p_t} > \frac{1}{2}$ then we will have case 2. Let $S_m$ be the set of all sub trees $t$ for which $q_m$ is the middle gap and case 3 only applies. We have that \\
$C \leq H + \sum b_t = H + 2 \sum\limits_{r = 1}^n p_r + 2 \sum\limits_{m = 0}^n q_m + \sum\limits_{m = 0}^n \sum\limits_{t \in S_m} \frac{4}{5}\frac{q_m^2}{p_t}$ \\ 
By factoring out $q_m$ and examining only cases up to $\frac{q_m}{p_t} = \frac{1}{2}$ we get: \\
$\implies C \leq H + 2 + \sum\limits_{m = 0}^n \frac{4}{5} q_m \sum\limits_{x=0}^{\infty} \frac{1}{2} * (\frac{4}{5}) ^ x$ \\
$\implies C \leq H + 2 + \sum\limits_{m = 0}^n \frac{4}{5}q_m (\frac{1}{2} * \frac{1}{1-\frac{4}{5}})$ (geometric series) \\
$\implies C \leq H + 4$ \\

\end{proof}

\chapter{Approximate Binary Search in the Hierarchical Memory Model} 
 
\section{The Hierarchical Memory Model}

The Hierarchical Memory Model (HMM) was proposed in 1987 by Aggarwal et al. as an alternative to the classic RAM model \cite{aggarwal1987model}. It was intended to better model the multiple levels of the memory hierarchy. The model has an unlimited number of registers, $R_1, R_2, ...$ each with its own location in memory (positive integer). In the first version of the model, accessing a register at memory location $x_i$ takes $\lceil lg(x_i) \rceil$ time. Thus, computing $f(a_1, a_2, ..., a_n)$ takes time $\sum_{i=1}^{n} \lceil lg(location(a_i)) \rceil$. The original paper also considered arbitrary cost functions $f(x)$. We will use the cost function as was explained in Thite's thesis \cite{thite2008optimum}. Here, $\mu (a)$ is the cost of accessing memory location $a$. We have a series of memory sizes $m_1, m_2, ..., m_h$ where $m_l$ has infinite size each with an associated cost $c_1, c_2, ..., c_h$. We assume that $c_1 < c_2 < ... < c_h$. 

\begin{center}$\mu (a) = c_i if \sum_{j = 1}^{i-1}m_j  < a \leq \sum_{j = 1}^{i}m_j$. \end{center}

While Thite notes that typical memory hierarchies have increasing sizes for slower memory levels, we explicitly assume that successive memory sizes divide one another evenly:
\begin{center}
$m_1 < m_2 < ... < m_h$ and
\end{center}
\begin{center}
$\forall i \in  \{1,2,...,h\}$, $m_i \mid m_{i+1}$.
\end{center}

\section{Thite's Optimum Binary Search Trees on the HMM Model}

Thite's thesis provided solutions to several problems in the HMM and a related model \cite{thite2008optimum}. He first provided an optimal solution to the following problem (known as \textbf{Problem 5} in the work):\\


\textbf{Problem [Optimum BST Under HMM].} Suppose we are given a set of $n$ ordered keys $x_1, x_2, ..., x_2$ with associated probabilities of search $p_1, p_2, ..., p_n$, as well as $n+1$ ranges $(- \infty, x_1), (x_1, x_2), ..., (x_{n-1}, x_n), (x_n, \infty)$ with associated probabilities of search $q_0, q_1, ..., q_n$. The problem is to construct a binary search tree $T$ over the set of keys and compute a memory assignment function $\phi : V (T) \rightarrow {1, 2, ..., n}$ that assigns nodes of $T$ to memory locations such that the expected cost of a search is minimized under the HMM model.\\


He provided three separate optimum solutions; \textbf{Parts}, \textbf{Trunks}, and \textbf{Split}. These algorithms have various use cases, running in times $O(\frac{2^{h-1}}{(h-1)!}* n^{2h+1})$, $O(\frac{2^{n-m_h}*(n-m_h+h)^{n-m_h}*n^3}{(h-2)!})$ and $O(2^n)$ respectively. In the following sections, I will provide an approximate solution to this problem that runs in time $O(n*lg(m_1))$ and an upper bound on its expected search cost.  \\

Thite also considered the same problem under the related HMM$_2$ model. This model assumes there are simply two levels of memory of size $m_1$ and $m_2$ with costs of access $c_1$ and $c_2$ where $c_1 < c_2$. Thite provided an optimal solution to this problem (named \textbf{TwoLevel}) which ran in time $O(n^5)$, $o(n^5)$ if $m_1 \in o(n)$, and $O(n^4)$ if $m_1 \in O(1)$. He also gave an $O(nlg n)$ time approximate solution with an upper bounded expected search cost of $c_2(H+1)$. The solution I provide which approximates the optimum tree under the HMM model also provides a strict improvement over Thite's approximate algorithm in both running time and expected cost under the HMM$_2$ model.


\section{Efficient Near-Optimal Multiway Trees of Bose and Dou\"{i}eb}
In 2009, Bose and Dou\"{i}eb's 2009 work provided a new construction method with linear running time (independent of the size of a node in tree) with the best expected cost to date \cite{bose2009efficient}. The group was able to prove that:
\begin{center}
$\frac{H}{lg(2k-1)} \leq P_{OPT} \leq P_T \leq \frac{H}{lg k} + 1 + \sum_{i=0}^n q_i - q_0 - q_n - \sum_{i=0}^m q_{rank[i]}$
\end{center}
Here, H is the entropy of the probability distribution, $P_{OPT}$ is the average path-length in the optimal tree, $P_T$ is the average path length of the tree built using their algorithm and $m=max({n-3P,P})-1 \geq \frac{n}{4} - 1$. $P$ is the number of increasing or decreasing sequences in a left-to-right read of the access probabilities of the leaves. Moreover, $q_{rank[i]}$ is the $i^{th}$ smallest access probability among all leaves except $q_0$ and $q_n$. \\
 As described in the subsection TODO, we are given $n$ ordered keys with weights $p_0, ..., p_n$ as well as $n+1$ weights of unsuccessful searches $q_0,...,q_n$. We often refer to "keys" representing the gaps as $(x_{i-1},x_i)$ to be mean the "key" associated with a search between key $x_{i-1}$ and $x_i$. \\
 
 The algorithm occurs in three steps. First, a new distribution is created by distributing all leaf weight to internal nodes. The peaks and valleys of the probability distribution of the leaf weights is used to do this assignment. After this, the new probability distribution is made into a k-ary tree using a greedy recursive algorithm. The algorithm essentially chooses the $l \leq k$ elements to go in the root node such that each child's subtree will have probability of access of at most $1/k$. They then reattach leaves to this internal key tree they have created. Their algorithm's design allows them to bound the depth of keys and leaves which ultimately allows them to achieve the bounds they have described. 

\section{Algorithm ApproxMWPaging}

First, we need a lemma about converting a multiway search tree to a binary search tree. For the sake of clarity, we will call what are typically known as \textit{nodes} of the multiway tree \textit{pages}. This represents how various items of our search tree will fit onto pages. We maintain the notion of calling individual items \textit{keys}.

\begin{lem}
Given a multiway tree $T'$ with page size $k$ and $n$ keys, where keys are associated with a probability distribution of successful and unsuccessful searches as in Knuth's original optimum binary search tree problem, we can create a BST $T$ where each key in a given page $g$ of $T'$ forms a connected component in $T$ in $O(nlg(k))$ time.
\end{lem}

\begin{proof}
For each page $g$, we sort its keys and create a complete BST $B$ over the keys. We create an ordering over all potential locations where keys could be added to the tree from left to right. All keys in all descendant pages of a page in a specific subtree rooted at a child of $g$ will lie in a specific range. There are at most $k+1$ of these ranges (since our page has at most $k$ keys). These ranges will precisely correspond to the at most $k+1$ locations where a new child key could be added to $B$. We note that we cannot add new children to keys which correspond to unsuccessful searches. We order these locations from left to right and attach root keys from the newly created BST's of each of the ordered (left to right) children of $g$. These will all be valid connections since each child of $g$ has keys in these correct ranges, and combining BST's in this fashion produces a valid BST. We perform a sort of $O(k)$ items (in time $O(klg(k))$) $O(n/k)$ times, and make $O(n)$ new parent child connections, giving us total time $O(n lg(k)$.
\end{proof}

In order to approximately solve the optimum BST problem under the HMM model, we will do the following: \\

1) First, we create a Multiway Tree $T'$ using the algorithm of Bose and Dou\"{i}eb. This takes $O(n)$ time with our node size equal to $m_1$, the smallest level of our memory hierarchy \cite{bose2009efficient}. \\

2) Inside each page (node of the multiway tree), we create a balanced binary search tree (ignoring weights). We call each of these $T'_k$ for $k \in {1,...,\lceil n/m_1 \rceil}$.   This takes $O(m_1)$ time per page, of which there are at most $O(n/m_1)$ giving us $O(n)$ time total. \\

3) In order to make this into a proper binary search tree, we must connect the $O(n/m_1)$ BST's we have made as described in the previous Lemma. This takes $O(n lg(m_1))$ time. \\

4) We pack things into memory in a breadth first search order in $T$ starting from the root. This takes $O(n)$ time. \\


\textit{THIS IS THE OLD WAY. I THINK ITS NEEDED FOR TH BETTER MODEL BUT NOT FOR THIS. We pack things into memory in a breadth first search order in $T'$ starting from the root. We do not leave gaps if pages are not full. The individual items of the pages of size at most $m_1$ are stored in breadth first search order based on the trees made in step 2. This takes $O(n/m_1)$ for the breadth first search of $T'$, and $O(m_1)$ per page, of which there are $O(n/m_1)$, giving us $O(n)$ time.\\} 

We are left with a binary search tree which has been properly packed into our memory in total time $O(n lg(m_1))$.

\section{Expected Cost ApproxMWPaging}

First, we bound the depth of nodes in our BST $T$. The depth of a key $x_i$ (denoted $(x_{i-1}, x_i)$ for unsuccessful searches) is defined as $d_T(x_i)$ ($d_T(x_{i-1},x_i)$).

\begin{lem}
For a key $x_i$,
\begin{center} $d_T(x_i) \leq \lceil lg(m_1) \rceil * \lfloor log_{m_1}(\frac{1}{p_i}) \rfloor$. \end{center}
For a key $(x_{i-1},x_i)$,
\begin{center} $d_T(x_{i_1},x_i) \leq \lceil lg(m_1) \rceil * (\lfloor log_{m_1}(\frac{2}{q_i}) \rfloor + 1)$. \end{center}
\end{lem}

\begin{proof}
First we note that in the tree $T'$ we build using Bose and Dou\"{i}eb's multiway tree algorithm, the maximum depth of keys (call this $d_T'(x_i), d_T'(x_{i-1},x_i)$) for a page size $m_1$ is \cite{bose2009efficient}:
\begin{center} $d_T(x_i) \leq \lfloor log_{m_1}(\frac{1}{p_i}) \rfloor$. \end{center} 
\begin{center} $d_T(x_{i-1},x_i) \leq \lfloor log_{m_1}(\frac{2}{q_i}) \rfloor + 1$. \end{center}
As explained in the paper, these follow from Lemmas 1 and 2 of Bose and Dou\"{i}eb \cite{bose2009efficient}.

Within a page, we make a balanced (ignoring weight) BST, so each key has a depth \textit{within} a page of at most $\lceil lg(m_1) \rceil$.

Since our algorithm always connects the root of the BST made for page to a key in the BST made for the page's parent, a key $x_i$ has a depth of at most $\lfloor log_{m_1}(\frac{1}{p_i}) \rfloor$ in terms of pages accessed ($\lfloor log_{m_1}(\frac{2}{q_i}) \rfloor + 1$ pages for keys ($(x_{i-11},x_i)$). Within a page, we will examine at most $\lceil lg(m_1) \rceil$ keys. Thus, a key's depth is at most the bound described.
\end{proof}

\iffalse

Let $min_{p,q} = min(min_{i \in \{1, ..., n\}}(p_i), min_{j \in \{0, ..., n\}}(q_j))$ be probability of searching for the smallest weighted key (successful or unsuccessful search). The by the lemma above it immediately follows that:
\begin{cor}
\begin{center} $height(T) \leq d_T(min_{p,q}) \leq \lceil lg(m_1) \rceil * (\lfloor log_{m_1}(\frac{2}{min_{p,q}}) \rfloor + 1)$. \end{center}
\end{cor}

Using this, we can prove a more specific lemma about the height of our tree. 

\begin{lem}
 $height(T) \leq lg(\frac{1}{min_{p,q}}) + lg(m_1) + 2$.
\end{lem}

\begin{proof}
From our corollary we have that:\\
$height(T) \leq \lceil lg(m_1) \rceil * (\lfloor log_{m_1}(\frac{2}{min_{p,q}}) \rfloor + 1)$ \\

$ \leq \lceil lg(m_1) \rceil * (\lfloor \frac{lg(\frac{2}{min_{p,q}})}{lg(m_1)} \rfloor + 1)$\\

$ \leq \lceil lg(m_1) \rceil * ( \frac{ \lfloor lg(\frac{2}{min_{p,q}}) \rfloor}{\lceil lg(m_1) \rceil} + 1)$\\

$ \leq  \lfloor lg(\frac{2}{min_{p,q}}) \rfloor + \lceil lg(m_1) \rceil$\\

$ \leq  lg(\frac{1}{min_{p,q}}) + lg(m_1) + 2$.

\end{proof}

\fi

We now describe the cost of searching for a key located at deepest part of tree $T$: $W$. Let $m'_j = \sum_{k \leq j} m_j$. We define $m'_0 = 0$. Let $l$ be the smallest $j$ such that $m'_j > n$. $l$ represents how many levels of memory will be required for storing our tree. Let $H(T)$ represent the height of our tree.

\iffalse
\begin{lem} \hspace{1cm} \\
$W \leq \sum_{k=1}^{l-1} (\lfloor lg(m'_i+1) \rfloor - \lfloor lg(m'_{i-1}+1) \rfloor)*c_i$\\ $+ (lg(\frac{1}{min_{p,q}}) + lg(m_1) + 2 - \lfloor lg(m'_{l-1}+1) \rfloor)*c_l$ THIS USES THE OLD HEIGHT
\end{lem}
\fi

\begin{lem} 
$W \leq \sum_{k=1}^{l-1} (\lfloor lg(m'_i+1) \rfloor - \lfloor lg(m'_{i-1}+1) \rfloor)*c_i+$\\  $ (H(T) - \lfloor lg(m'_{l-1}+1) \rfloor)*c_l$
\end{lem}


\begin{proof}
Consider the accessing each key along the path from the root to the deepest key in $T$. Note that we will have to examine $H(T)$ keys in order to access the deepest leaf in the tree. We will access one key a depth $0$, one key at depth $1$ and so on. Because the tree is packed into memory in BFS order, a key a depth $i$ will be at memory location at most $2^i-1$. Now, consider how many levels of the binary search tree $T$ will fit inside of $m_1$, the fastest memory. In order for all keys of depth $i$ (and higher) to be in $m_1$ we need: \\
$2^i-1 \leq m_1 \implies i \leq lg(m_1 + 1)$. \\
Thus, at least $\lfloor lg(m_1 + 1) \rfloor$ levels of $T$ fit on $m_1$. Next we examine how many fit on $m_j$ for $1 < j < l$. The last level to completely fit on $m_j$ or higher memories is the maximum $i$ such that: \\
$2^{i+1}-1 \leq m'_2 \implies i \leq lg(m'_j + 1)$. \\
Thus, at least $\lfloor lg(m'_j + 1) \rfloor$ levels of $T$ fit on $m_j$ or higher levels of memory. \\
Thus, on our search for the deepest key in the tree, we will make at least $\lfloor lg(m_1 + 1) \rfloor$ checks for elements located at memory $m_1$. This will cost a total of\\
\begin{center}
$\lfloor lg(m_1 + 1) \rfloor * c_1$.\\
\end{center}
For each memory level $m_j$ for $1 < j < l$, we will make at least $\lfloor lg(m'_j + 1) \rfloor$ checks in $m_j$ or higher memories. Of these checks, at least $\lfloor lg(m'_{j-1} + 1) \rfloor$ will be in memory levels strictly higher up in the memory hierarchy than $j$. Since $c_j > c_i$ for $j > i$, an upper bound on the cost of searching for all elements in memory level $j$ on the path from the root to the deepest element of the tree is: \\
\begin{center}
$(\lfloor lg(m_1 + 1) \rfloor - \lfloor lg(m'_{j-1} + 1) \rfloor) * c_j$.\\
\end{center}
Finally, we can upper bound the cost by assuming that all remaining searches take place at $c_l$. The searches at level $l$ will cost at most: \\

\iffalse
\begin{center}
$(lg(\frac{1}{min_{p,q}}) + lg(m_1) + 2 - \lfloor lg(m'_{l-1}+1) \rfloor)*c_l$. \\
\end{center}
\fi

\begin{center}
$(lg(H(T) - \lfloor lg(m'_{l-1}+1) \rfloor)*c_l$. \\
\end{center}

Combining the above three equations gives the desired bound.

\end{proof}

Next, we note that the cost of search for a key at some depth $i$ is at most $\frac{i}{H(T)}* W$.

\begin{lem}
The cost of searching for a keys $x_i$ and $(x_{i-1},x_i)$ ($C(x_i)$ and $C(x_{i-1},x_i)$ respectively)can be bounded as follows: 

\begin{center} $C(x_i) \leq \frac{\lceil lg(m_1) \rceil * \frac{lg(\frac{1}{p_i})}{lg(m_1)}}{\lceil lg(2n+2) \rceil} * W$ 
\end{center}


\begin{center}
 $C(x_{i-1},x_i) \leq \frac{\lceil lg(m_1) \rceil * \lfloor log_{m_1}(\frac{2}{q_i}) \rfloor}{\lceil lg(2n+2) \rceil} * W$ 
 \end{center}
\end{lem}

\begin{proof}
From Lemma 4.5.1 we have a bound on the depth of keys $x_i$ and $(x_{i-1},x_i)$. Moreover, we know that since there are $2n+1$ keys in our tree, our tree must have a height of at least:
\begin{center}
 $\lceil lg(2n+2) \rceil$.
\end{center}
Note that since our tree is stored in BFS order in memory, whenever we examine a key's child, it will be at a memory location of at least the same, if not higher cost (by being in the same or a deeper page). Thus, the path from the root to a specific key can be upper bounded by the cost of searching for the deepest element in the tree, multiplied by the fraction of the way down the tree $x_i$ or $(x_{i-1},x_i)$ are. Note that when searching for keys, we must search along the entire path from root to the key in question, while we need only examine the path from the root to the parent of a key for unsuccessful $(x_{i-1},x_i)$ searches. Combining 4.5.1 and 4.5.4 with the minimum height of our tree gives: \\

 $C(x_i) \leq \frac{\lceil lg(m_1) \rceil * \lfloor log_{m_1}(\frac{1}{p_i}) \rfloor}{\lceil lg(2n+2) \rceil} * W$  \\
 $\implies C(x_i) \leq \frac{\lceil lg(m_1) \rceil * \frac{lg(\frac{1}{p_i})}{lg(m_1)}}{\lceil lg(2n+2) \rceil} * W$ \\
 
 Similarly, \\
 $C(x_{i-1},x_i) \leq \frac{\lceil lg(m_1) \rceil * \lfloor log_{m_1}(\frac{2}{q_i} + 1 - 1) \rfloor}{\lceil lg(2n+2) \rceil} * W$  \\
 $C(x_{i-1},x_i) \leq \frac{\lceil lg(m_1) \rceil * \frac{lg(\frac{2}{q_i})}{lg(m_1)}}{\lceil lg(2n+2) \rceil} * W$. \\ 
 
This completes the proof. 
 


\end{proof}

We can now bound the expected cost of search using the bounds for each key.

\begin{thm}
$C \leq  (\frac{\lceil lg(m_1) \rceil}{\lceil lg(2n+2) \rceil*lg(m_1)} * W) * (H + 1)$
\end{thm}

\begin{proof}
The total expected cost of search is simply the sum of the weighted cost of search for all keys. Given our last lemma, we have that:

$C \leq \sum_{i=1}^{n} p_i*C(x_i) + \sum_{j=1}^{n+1} q_i*C(x_{i-1},x_i)$ \\

$\implies C \leq \sum_{i=1}^{n} p_i*\frac{\lceil lg(m_1) \rceil * \frac{lg(\frac{1}{p_i})}{lg(m_1)}}{\lceil lg(2n+2) \rceil} * W + \sum_{j=1}^{n+1} q_i*\frac{\lceil lg(m_1) \rceil * \frac{lg(\frac{2}{q_i})}{lg(m_1)}}{\lceil lg(2n+2) \rceil} * W$ \\

$\implies C \leq  (\frac{\lceil lg(m_1) \rceil}{\lceil lg(2n+2) \rceil*lg(m_1)} * W) * (\sum_{i=1}^{n} p_i*lg(\frac{1}{p_i}) + \sum_{j=1}^{n+1} (q_i*lg(\frac{2}{q_i})))$ \\

$\implies C \leq  (\frac{\lceil lg(m_1) \rceil}{\lceil lg(2n+2) \rceil*lg(m_1)} * W) * (\sum_{i=1}^{n} p_i*lg(\frac{1}{p_i}) + \sum_{j=1}^{n+1} (q_i*lg(\frac{1}{q_i}))+ 1$ \\

$\implies C \leq  (\frac{\lceil lg(m_1) \rceil}{\lceil lg(2n+2) \rceil*lg(m_1)} * W) * (H + 1)$.
\end{proof}


\section{Approximate Binary Search Trees of De Prisco and De Santis with Extensions by Bose and Dou\"{i}eb} \label{sec:deBST}

As in the classic Knuth problem, we are given a set of $n$ probabilities of searching for words ($p_1, p_2, ..., p_n$), as well as $n+1$ probabilities of unsuccessful searches ($q_0, q_1, ..., q_n$). The duo provides an algorithm which constructs a binary search tree in $O(n)$ time with an expected cost of at most \cite{de1993binary}
\begin{center}
$H+1-q_0-q_n+q_{max}$
\end{center}  where $q_{max}$ is the maximum probability of an unsuccessful search. This was later modified by Bose and Dou\"{i}eb (the same paper described in section TODO) to have an improved bound \cite{bose2009efficient}
\begin{center}
$P_T \leq H + 1 - q_0 - q_n + q_{max} - \sum_{i=0}^{m'} pq_{rank[i]}$
\end{center}
$P_T$ is the average path length of the tree built using their algorithm and $m'=max({2n-3P,P})-1 \geq \frac{n}{2} - 1$. $P$ is the number of increasing or decreasing sequences in a left-to-right read of the access probabilities of the leaves. Moreover, $pq_{rank[i]}$ is the $i^{th}$ smallest access probability among all keys and leaves except $q_0$ and $q_n$. \\

First, I explain the algorithm of De Prisco and De Santis and then explain the extensions of Bose and Dou\"{i}eb. De Prisco and De Santis' algorithm occurs in three phases.\\

In \textbf{Phase 1}, an auxiliary probability distribution is created using $2n$ zero probabilities, along with the $2n+1$ successful and unsuccessful search probabilities. Yeung's linear time alphabetic search tree algorithm is used with the $2n+1$ successful and unsuccessful search probabilities used as leaves of the new tree created. This is referred to as the \textit{starting tree}. \\

In \textbf{Phase 2} what's known as the \textit{redundant tree} is created by moving $p_i$ keys up the \textit{starting tree} to the lowest common ancestor of keys $q_{i-1}$ and $q_i$. The keys which used to be called $p_i$ are relabelled to $old.p_i$. \\

In \textbf{Phase 3} the \textit{derived tree} is constructed from the \textit{redundant tree} by removing redundant edges. Edges to and from nodes which represented zero probability keys are deleted. This derived tree is a binary search tree with the expected search cost described. \\

In Bose and Dou\"{i}eb's work, they explain how they can substitute their algorithm for Yeung's linear time alphabetic search tree algorithm which results in a better bound (as described above). We use the updated version (by Bose and Dou\"{i}eb) of De Prisco and De Santis' algorithm as a subroutine in the sections to follow.

\section{Algorithm ApproxBSTPaging}

Our second solution to create an approximately optimum BST under the HMM model works as follows: \\

1) First, we create an approximately optimal BST $T$ using the algorithm of De Prisco and De Santis \cite{de1993binary} (as updated by Bose and Dou\"{i}eb \cite{bose2009efficient}). This takes $O(n)$ time. \\

2) In a similar fashion to step 4) of \textit{ApproxMWPaging}, we pack things into memory in a breadth first search order of $T$ starting from the root. This relative simple traversal also takes $O(n)$ time. \\

We are left with a binary search tree which has been properly packed into our memory in total time $O(n)$.


\section{Expected Cost ApproxBSTPaging}

As explained in the Bose and Dou\"{i}eb paper, the average path length search cost of the tree created by their algorithm is at most:\\ \cite{bose2009efficient}
$P_T \leq H + 1 - q_0 - q_n + q_{max} - \sum_{i=0}^{m'} pq_{rank[i]}$ \\

Similar to the previous section, if we can bound the cost of search for a given path length, then we can form a bound on the average cost of search in the HMM model. Consider $min_{p,q}$ for our probability distribution. It is implicitly shown in the work of Bose and Dou\"{i}eb that: \\

\iffalse
Using this, we can prove a more specific lemma about the height of our tree. 

\begin{lem}
 $H(T) \leq  depth(min_{p,q}) \leq \lfloor lg(\frac{1}{min_{p,q}}) \rfloor + 2$.
\end{lem}

\begin{proof}
I will simply give an outline here as the bulk of the proof follows from statements of Bose and Dou\"{i}eb, and is extremely similar to the ones presented in the previous sections. After \textbf{phase 1} of the algorithm, all "leaves" ($p's$ or $q's$ from our original data set) with probability $pq_i$ have depths of at most $\rfloor lg(\frac{1}{pq_i} \lfloor + 2$. After the final two phases of algorithm, at most one leaf has not moved up the tree. Thus, $depth(min_{p,q}) \leq lg(\frac{1}{min_{p,q}}) + 2$. Since $min_{p,q}$ is the smallest element in the distribution, this gives an upper bound on the height of the whole tree and the result follows.
\end{proof}

\fi


As before, we can describe the cost of searching for a key located at deepest part of tree $T$: $W$. Recall, $m'_j = \sum_{k \leq j} m_j$, $m'_0 = 0$ and let $l$ be the smallest $j$ such that $m'_j > n$.

\begin{lem} \hspace{1cm} \\
$W \leq \sum_{k=1}^{l-1} (\lfloor lg(m'_i+1) \rfloor - \lfloor lg(m'_{i-1}+1) \rfloor)*c_i$\\ $+ (H(T) - \lfloor lg(m'_{l-1}+1) \rfloor)*c_l$ 
\end{lem}

\begin{proof}
Since we are simply putting things into memory in BFS order, and all we use is the height of the tree and the memory hierarchy, the proof is identical to that of \textbf{Lemma 4.5.3}.
\end{proof}

Next, we note that the cost of search for a key at some depth $i$ is at most $\frac{i}{H(T)}* W$. This must be true because the cost of examining nodes can only increase as we move down the tree because of the increasing property of the cost of access of our various levels of memory and our BFS packing algorithm. Since, we also know the expected path length of our tree we can show that: \\

\begin{thm}
$C \leq  (\frac{W}{H(T)}) * (H + 1 - q_0 - q_n + q_{max} - \sum_{i=0}^{m'} pq_{rank[i]})$
\end{thm}

\begin{proof}

Bose and Dou\"{i}eb show that after using their algorithm for Phase 1 of De Prisco and De Santis algorithm algorithm, every leaf of the \textit{starting tree} (all keys representing successful and unsuccessful searches) are at depth at most $\lfloor lg(\frac{1}{p}) \rfloor + 1$  for at least $\{2n-3P,P\}-1$ of $p \in \{ \{p_1, p_2, ..., p_n \} \cup \{ q_0, q_1, ..., q_n \} \}$ and $\lfloor lg(\frac{1}{p}) \rfloor + 2$ for all others. Recall that $P$ is the number of peaks in the probability distribution $q_0, p_1, q_1, ..., p_n, q_n$. Moreover, after phases two and three of the algorithm, each key has its depth decrease by 2, and all leaves (except one) move up the tree by one. Now, let $C_d(i)$ represent the cost of accessing a node at depth $i$. As mentioned above, this is at most $\frac{i}{H(T)}* W$. Using this, we can derive the following equation an upper bound on the expected cost of search: \\

$C \leq \sum_{i=1}^{n}(p_i*\frac{depth(p_i)}{H(T)}* W)+ \sum_{i=0}^{n}(q_i*\frac{(depth(q_i)-1)}{H(T)}* W)$ \\

$\implies C \leq \frac{W}{H(T)}(\sum_{i=1}^{n}(p_i*depth(p_i))+ \sum_{i=0}^{n}(q_i*(depth(q_i)-1)))$ \\

$\implies C \leq \frac{W}{H(T)}(\sum_{i=1}^{n}(p_i*depth(p_i))+ \sum_{i=0}^{n}(q_i*(depth(q_i)-1)))$ \\

$\implies C \leq \frac{W}{H(T)}(\sum_{i=1}^{n}(p_i*\lfloor lg(\frac{1}{p_i}) \rfloor+ \sum_{i=0}^{n}(q_i*\lfloor lg(\frac{1}{q_i}) -1 + 1 \rfloor) )$ \\

Note that the expected cost is equal to the expected path length for each key, multiplied by the cost of search for each key. Since we have an upper bound on the cost of search for an element at some depth $i$, and an upper bound on the expected path length TODO
\end{proof}

IS IT DEPTH + 1 FOR THE KEYS?
SHOULD WE BE DOING AN EXTRA ACCESS FOR LEAVES RELATIVE TO HOW THEY DID IT BEFORE.


\section{Improvements over Thite in the HMM$_2$ Model}

The HMM$_2$ model is the same as the general HMM model with the added constraint that there are only two types of memory (slow and fast). In Thite's thesis, he proposed both an optimal solution to the problem, as well as an approximate solution that runs in time $O(n lg(n))$ \cite{thite2008optimum}. I show that my solution is at least as good, and may be better in certain situations. First I give an alternate bound to my expected cost which is similar to the one presented in Thite's work.

\begin{lem}
Let $m'_j = \sum_{k \leq j} m_j$. We define $m'_0 = 0$. Let $l$ be the smallest $j$ such that $m'_j > n$. Then the cost of search: \\
$C \leq  c_l * (H + 1 - q_0 - q_n + q_{max} - \sum_{i=0}^{m'} pq_{rank[i]})$.
\end{lem}

\begin{proof}
By using the ApproxBSTPaging algorithm, we can simply assume all key accesses happen in the most expensive type of memory ($m_l$). Thus, our expected cost can be bounded by an upper bound on the average path length, multiplied by this upper bound on our cost of accessing each item.
\end{proof}

Moreover, for both ApproxMWPaging and ApproxBSTPaging:\\
\begin{lem}
Suppose $\sum_{k=1}^{l-1} (c_{i+1}-c_i)*(\lfloor lg(m'_i+1) \rfloor) > \lfloor lg(m'_1+1) \rfloor*c_1$. Then \\
 $W < c_l*H(T)$. 
\end{lem}

\begin{proof}
$W \leq \sum_{k=1}^{l-1} (\lfloor lg(m'_k+1) \rfloor - \lfloor lg(m'_{k-1}+1) \rfloor)*c_i$\\ $+ (height(T) - 1 - \lfloor lg(m'_{k-1}+1)-1 \rfloor)*c_l$ \\
Rearranging the formula gives: \\
$\implies W \leq H(T)*c_l + \lfloor lg(m'_1+1) \rfloor - \sum_{k=1}^{l-1} (c_{k+1}-c_k)\lfloor lg(m'_k+1) \rfloor $\\
Thus, as long as $\sum_{k=1}^{l-1} (c_{k+1}-c_k)\lfloor lg(m'_k+1) \rfloor > \lfloor lg(m'_1+1) \rfloor$ then we will get our desired result.

\end{proof}

Note that our assumption will generally be true since $c_l > c_{l-1} > ... > c_1$ and $m_l > m_{l-1} > ... > m_1$. Finally, we can see that: \\

\begin{thm}
Suppose $\sum_{k=1}^{l-1} (c_{i+1}-c_i)*(\lfloor lg(m'_i+1) \rfloor) > \lfloor lg(m'_1+1) \rfloor*c_1$. Then \\
$C <  c_l * (H + 1 - q_0 - q_n + q_{max} - \sum_{i=0}^{m'} pq_{rank[i]})$
\end{thm}

\begin{proof}
$C \leq  (\frac{W}{H(T)}) * (H + 1 - q_0 - q_n + q_{max} - \sum_{i=0}^{m'} pq_{rank[i]})$
Assuming $\sum_{k=1}^{l-1} (c_{i+1}-c_i)*(\lfloor lg(m'_i+1) \rfloor) > \lfloor lg(m'_1+1) \rfloor*c_1$ and using \textbf{Lemma 4.9.2} gives: \\
$\implies C <  (\frac{c_l*H(T)}{H(T)}) * (H + 1 - q_0 - q_n + q_{max} - \sum_{i=0}^{m'} pq_{rank[i]})$ \\
$\implies C <  c_l * (H + 1 - q_0 - q_n + q_{max} - \sum_{i=0}^{m'} pq_{rank[i]})$
\end{proof}

TODO UNDERSTAND WHY THIS ISN'T A DIRECT IMPROVEMENT. I'M CONFUSED

Subbing our assumption into the HMM$_2$ model gives:
$(c_2-c_1)*(\lfloor lg(m_1+1) \rfloor) > \lfloor lg(m_1+1) \rfloor*c_1$ \\
$\implies c_2 > 2*c_1$ \\
which is a valid assumption in many types of memory hierarchies.

\begin{cor}
$C <  c_2 * (H + 1 - q_0 - q_n + q_{max} - \sum_{i=0}^{m'} pq_{rank[i]})$ \\ whenever $c_2 > 2*c_1$
\end{cor}

In many types of memory hierarchies, this will provide an improvement over Thite's bound of: \\
$C < c_2 * (H+1)$ \\
especially in the case where $c_2 >> c_1$.




\chapter{Approximate Binary Search in the Hierarchical Memory with Block Transfer Model}

\section{Hierarchical Memory with Block Transfer Model}
This model (HMBTM for short) was proposed by Aggarwal, Chandra, and Snir as an improvement over the HMM model \cite{aggarwal1987model} discussed previously \cite{aggarwal1987hierarchical}. The model with block transfer still has a memory hierarchy with increasing costs of access, but allows for any contiguous block of memory to be copied from one location to another with constant unit time per element after the initial access. This provides a better model for standard computers which can copy many words from one type of memory to another after accessing a single word. Like the HMM model, we have an unlimited number of registers (numbered 1, 2, , .. etc.) and we will still use the cost function as was explained in Thite's thesis \cite{thite2008optimum}. Here, $\mu (a)$ is the cost of accessing memory location $a$. We have a series of memory sizes $m_1, m_2, ..., m_h$ where $m_l$ has infinite size each with an associated cost $c_1, c_2, ..., c_h$. We assume that $c_1 < c_2 < ... < c_h$. 

\begin{center}$\mu (a) = c_i if \sum_{j = 1}^{i-1}m_j  < a \leq \sum_{j = 1}^{i}m_j$. \end{center}

We explicitly assume that successive memory sizes divide one another evenly:
\begin{center}
$m_1 < m_2 < ... < m_h$ and
\end{center}
\begin{center}
$\forall i \in  \{1,2,...,h\}$, $m_i \mid m_{i+1}$.
\end{center}

In this new model, we are given the additional operational opportunity to move a block from one location to another. Specifically, a block copy operation is defined as: \\
 $[x-l, x] \rightarrow [y-l,y]$. \\
The contents of $\mu(x-i)$ are copied to $\mu(y-i)$ for $i \in \{0,...,l\}$. This is valid if the two intervals are disjoint. This copy operation costs $max(f(x,y)) + l$.\\

I provide fast but approximate solutions to the optimum BST problem under new model in the following sections.

\section{ApproxMWPaging with BT}

The first algorithm I provide is very similar to ApproxMWPaging. The first three steps of the algorithm are in fact identical. To review, we create a Multiway Tree $T'$ using the algorithm of Bose and Dou\"{i}eb, form balanced BST's within each page of $T'$, and then connect these balanced BSTs from $T'$ to form a valid BST $T$. This is done in $O(n lg(m_1))$ time. \\

To decide packing order, we will do a modified BFS of $T'$. We will always pack individual pages of $T'$ into memory in a BFS of the balanced BST formed from the page. \\

BT\textendash Range-Pack packs the tree $T$ using $r\_q$ as roots from memory location $loc$. It packs recursively into pages of size $m_i$ (or smaller if necessary) until $m_{i+1}$ nodes have been packed or we run out of roots to pack from in $r\_q[]$. If $i=0$ then we only pack $amt$ nodes (so we can fill memory levels appropriately). Note that we only remove the top root from $r\_q$ if the root has been completely used i.e. it has been packed into a page of size $\leq m_1$. We always return a pair, the updated $r\_q$ which contains new roots to search packed from (added in BFS order from leaves of $m_1$ or smaller packed "pages") and the current location in memory where we should pack into next. \\

We call BT\textendash Pack() in order to pack nodes into the tree. Essentially, this function attempts to pack each level of the memory hierarchy independently from $m_1$ to $m_l$. Within each memory level $i$, we attempt to pack it from a single source (the next available node in BFS order of our tree) in blocks of size  $m_{i-1}$ which are recursively packed in blocks of size $m_{i-2}$. If we cannot fill a memory level, or a specific block within the packing of a memory level, we will pack whatever we can starting again from a new root (still in BFS order) passing into our BT\textendash Range-Pack function the biggest $i$ possible such that $m_i$ will still fit in the page we are trying to pack. \\

We are left with a binary search tree which has been properly packed into our memory in total time $O(n * l)$.


\begin{algorithm}[H]
\caption{ApproxPaging with BT Packing}
\label{AMWBTPack}
\begin{algorithmic}[1]
\Procedure{BT\textendash Range-Pack}{$r\_q[], i, loc, amt$}
\If{$i = 0$}
	\State Create largest BFS tree $T$ possible from $r\_q[0]$ with at $min(m_1, amt)$ nodes
	\State Push $T$ in BFS order into memory starting from memory location $loc$
	\State $loc += size(T)$
	\State $r\_q.pop()$
	\For{each leaf $l$ in $T$ in BFS order}
		\State Push all children of $l$ onto $r\_q$
	\EndFor
	\State \textbf{return} $(r\_q[], loc)$		
\Else
	\State $total = 0$
	\While{$total < m_{i+1}$ and $\neg (empty?$ $r)$}
		\State $old\_loc = loc$
		\State $i = max_{0,...,j+1}(i : m_i + total < m_{j+1})$
		\State $(loc, new\_r\_q) =$ AMWBT\textendash Range-Pack $(r\_q[0,...,0], i-1, loc, m_i)$
		\For{each node $r$ in $new\_r\_q$}
			\State $r\_q.push(r)$
		\EndFor
		\State $total += loc-old\_loc$
	\EndWhile
	\State \textbf{return} $(r\_q[], loc)$
\EndIf
\EndProcedure
\end{algorithmic}
\end{algorithm}

\begin{algorithm}[H]
\caption{ApproxPaging with BT Packing}
\label{AMWBTPack}
\begin{algorithmic}[1]
\Procedure{BT\textendash Pack}{}
\State $r\_q = empty$
\State $r\_q.push\_back(root)$
\State $loc = 0$
\For{$ j = 0$ to $l-1$}
	\While{$loc < m_{j+1}$ and $\neg(empty? r\_q)$}
		\If{$loc + m_1 < m_{j+1}$}
			\State $i = max_{0,...,j+1}(i : m_i + total < m_{j+1})$
			\State $(loc, new\_r\_q) =$ AMWBT\textendash Range-Pack $([r\_q[0]], i-1, loc, m_i)$		
		\Else
			\State $(loc, new\_r\_q) =$ AMWBT\textendash Range-Pack $([r\_q[0]], 0, loc, m_i)$
		\EndIf
		\State $r\_q.pop()$
		\For{each node $r$ in $new\_r\_q$}
			\State $r\_q.push(r)$
		\EndFor
	\EndWhile
\EndFor
\EndProcedure
\end{algorithmic}
\end{algorithm}


\section{ApproxMWPaging with BT Running Time }

First, we examine the runtime of BT\textendash Range-Pack$(r\_q[], i, loc, amt)$.

\begin{lem}
BT\textendash Range-Pack$(r\_q[], i, loc, amt)$ returns an updated location, call it $new\_loc$ in time at most $O((i+1)*(new\_loc - loc))$ time and packs $(new\_loc - loc)$ nodes into memory.
\end{lem}

\begin{proof}
I will prove this by induction on $i$. In the base case ($i=0$) we simply do a BFS from the given root, packing as many of $min(m_1, amt)$ into memory as possible, we return an updated $loc$ which is exactly equal to the size of the BFS tree we have created. We also push all children of all leaves of this BFS tree onto our $r\_q$. This all takes time $(new\_loc - loc) \in O(1*(new\_loc - loc))$ as required. \\

Suppose BT\textendash Range-Pack$(r\_q[], i, loc, amt)$ returns $new\_loc$ in time at most $O(l*(new\_loc - loc))$ time and packs $(new\_loc - loc)$ nodes into memory for all $i < k$. Consider BT\textendash Range-Pack$(r\_q[], k, loc, amt)$. Since $i \neq 0$ we enter the else case on line $11$. Now, we may call BT\textendash Range-Pack a number of times, but the sum of $loc$ updates will be at most $m_{i+1}$. By our induction hypothesis, we know each BT\textendash Range-Pack$(r\_q[], i, loc, amt)$ for $i < k$ packs sum number $(new\_loc - loc)$ of nodes into memory in time $O((i+1)*(new\_loc - loc))$. Moreover, other than the recursive calls, all lines take $O(1)$ except line $15$ which takes $O(i)$ and lines 17-18 which take $(new\_loc - loc)$ time since only a linear amount of children can be pushed onto $r\_q$ for each item packed into memory (and each child can only be pushed to $r\_q$ once, by its single parent). Let $S$ be the set of all calls $q$ BT\textendash Range-Pack$(r\_q[], q_i, loc, amt)$ is called directly from our original BT\textendash Range-Pack$(r\_q[], k, loc, amt)$ call and packs $q\_amt$ items into memory. Let $T_{RP}(i, new\_loc - loc)$ be the cost of BT\textendash Range-Pack$(r\_q[], i, loc, amt)$ which packs $new\_loc - loc$ items into memory. The we have that: \\

$T_{RP}(k, new\_loc - loc) = O(new\_loc - loc) + O(i) + \sum_{q \in S} T_{RP}(q_i, q\_amt)$ \\ 
By our induction hypothesis we have that: \\
$\implies T_{RP}(k, new\_loc - loc) \in O(new\_loc - loc) + O(i) + \sum_{q \in S} O(q_i*q\_amt)$ \\
Since $q_i$ is at most $i$ we have: \\
$\implies T_{RP}(k, new\_loc - loc) \in O(new\_loc - loc) + O(i) + i* \sum_{q \in S} q\_amt$ \\
Finally, since $new\_loc - loc = \sum_{q \in S} q\_amt$: \\
$\implies T_{RP}(k, new\_loc - loc) \in O(new\_loc - loc) + O(i) + i*(new\_loc - loc)$ \\
$\implies T_{RP}(k, new\_loc - loc) \in O((new\_loc - loc)*(i+1))$ as required.
\end{proof}

This leads us to our runtime for BT\textendash Pack(). 

\begin{lem}
BT\textendash Pack packs all nodes in our tree $T$ into memory in time $O(n*l)$.
\end{lem}

\begin{proof}

First note that, as in BT\textendash Range-Pack, we always update $loc$ to be the new location where should pack nodes into memory. Lines $5-18$ loop through each memory level. For a given level, we pack as much as can into it, using as large BT\textendash Range-Pack's as possible. Specifically, line $8$ takes $O(l)$ and lines $14-15$ take $(new\_loc - loc)$ time since only a linear amount of children can be pushed onto $r\_q$ for each item packed into memory (and each child can only be pushed to $r\_q$ once, by its single parent). Let $S_h$ be the set of all calls $q$ BT\textendash Range-Pack$(r\_q[], q_i, loc, amt)$ called from our BT\textendash Pack when $j = h$ (these each pack $q\_amt$ of nodes into memory). Let $packed\_amt$ be the amount packed by a call to BT\textendash Range-Pack in lines $9$ or $11$. By our previous proof, these take time $O((packed\_amt)*(i))$. Since $i$ is at most $l$ and $sum packed\_amt = n$ (we never pack anything twice) we have that the time required for BT\textendash Pack $T_P$ is: \\
$T_{P} \in \sum_{j=0}^{l-1})sum_{q \in S_j} O(q\_amt*j)$ \\
$\implies T_{P} \in o(l) * \sum_{j=0}^{l-1})sum_{q \in S_j} O(q\_amt)$ \\
$\implies T_{P} \in O(n*l)$ as required.

\end{proof}


\section{Search with ApproxMWPaging with BT}



\begin{algorithm}[H]
\caption{ApproxMWPaging with BT Search}
\footnotesize
\label{AMWBTPack}
\begin{algorithmic}[1]
\Procedure{AMWBT\textendash Search}{$mem, key, prev\_moved\_s, offset\_s$}
\If{$(\neg (empty? prev\_moved\_s)$ and $(mem >= prev\_moved\_s.top()))$ or $(mem \geq m_1)$}
	\If{$(\neg (empty? prev\_moved\_s)$ and $(mem >= prev\_moved\_s.top()))$}
		\State $mem += offset\_s.top()$
		\State $offset\_s.pop()$
		\State $prev\_moved\_s.pop()$
		\If{$neq (empty? offset\_s)$}
			\State $mem -= offset\_s.top()$
		\EndIf
	\EndIf
	\State $offset\_s.push(mem)$
	\State $i = min_{m'_i : m'_i > mem}$
	\State $amt\_move = m'_i - mem$
	\State $prev\_moved\_s.push(amt\_move)$
	\State $[mem, mem + amt\_move - 1] \rightarrow [0, amt\_move - 1]$
	\State AMWBT\textendash Search$(0, key, prev\_moved\_s, offset\_s)$

\Else 
	\State $offset = 0$
	\If{$\neg (empty? prev\_s\_moved)$ and $(mem >= prev_moved_s.top())$}
		\State $offset = offset\_s.top()$
	\EndIf	
	\If{$key = mem.key$}
		\State \textbf{return} $mem.value$
	\ElsIf{$key < mem.key$}
		\If{$mem$ has no left child}
			\State \textbf{return} $mem$
		\Else
			\State  AMWBT\textendash Search$(mem.left\_child - offset, key, prev\_moved\_s, offset\_s)$
		\EndIf
	\Else
		\If{$mem$ has no right child}
			\State \textbf{return} $mem$
		\Else
			\State  AMWBT\textendash Search$(mem.right\_child - offset, key, prev\_moved\_s, offset\_s)$
		\EndIf
	\EndIf	
\EndIf
\EndProcedure
\end{algorithmic}
\end{algorithm}

To run we call AMWBT\textendash Search$(0, key, empty, empty)$ since the memory location of the root should be $0$.




\section{Expected Cost ApproxMWPaging with BT}

How a path has cost
What cost is
similar arg to ApproxMWPaging


\section{ApproxBSTPaging with BT}

Given what we have already shown, this algorithm is extremely simple to explain. We first create a BST $T$ using the algorithm of De Prisco and De Santis \cite{de1993binary} (as updated by Bose and Dou\"{i}eb \cite{bose2009efficient}) in $O(n)$ time (as was the first step in ApproxBSTPaging). We then simply call BT\textendash Pack() in order to pack the tree into memory as described in the ApproxMWPaging with BT sections.


\section{Expected Cost ApproxBSTPaging with BT}
How a path has cost
What cost is
similar arg to ApproxMWPaging

\chapter{Conclusion and Open Problems}

\section{Discussion}

\section{Conclusion}



% The \appendix statement indicates the beginning of the appendices.
\appendix

% Add a title page before the appendices and a line in the Table of Contents
\chapter*{APPENDICES}
\addcontentsline{toc}{chapter}{APPENDICES}
%======================================================================
\chapter[PDF Plots From Matlab]{Matlab Code for Making a PDF Plot}
\label{AppendixA}


%----------------------------------------------------------------------
% END MATERIAL
%----------------------------------------------------------------------

% B I B L I O G R A P H Y
% -----------------------

% The following statement selects the style to use for references.  It controls the sort order of the entries in the bibliography and also the formatting for the in-text labels.
\bibliographystyle{plain}
% This specifies the location of the file containing the bibliographic information.  
% It assumes you're using BibTeX (if not, why not?).
\cleardoublepage % This is needed if the book class is used, to place the anchor in the correct page,
                 % because the bibliography will start on its own page.
                 % Use \clearpage instead if the document class uses the "oneside" argument
\phantomsection  % With hyperref package, enables hyperlinking from the table of contents to bibliography             
% The following statement causes the title "References" to be used for the bibliography section:
\renewcommand*{\bibname}{References}

% Add the References to the Table of Contents
\addcontentsline{toc}{chapter}{\textbf{References}}

\bibliography{uw-ethesis/uw-ethesis}
% Tip 5: You can create multiple .bib files to organize your references. 
% Just list them all in the \bibliogaphy command, separated by commas (no spaces).

% The following statement causes the specified references to be added to the bibliography% even if they were not 
% cited in the text. The asterisk is a wildcard that causes all entries in the bibliographic database to be included (optional).
\nocite{*}

\end{document}
